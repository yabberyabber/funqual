\begin{figure}
    \centering
    \begin{tikzpicture}[scale=1.0]
        \tikzset{vertex/.style = {shape=circle,draw,minimum size=8em}}
        \tikzset{edge/.style = {->,> = latex'}}
        % vertices
        \node[vertex] (main) at
            (0,0) {\lstinline{main}};
        \node[vertex] (savethepandas)
            at  (0,-5) {\lstinline{save_the_pandas}};
        \node[vertex] (stopdeforestation)
            at  (5,-5) {\lstinline{stop_deforestation}};
        \node[vertex] (breedandreleasepandas)
            at  (5,-10) {\lstinline{breed_and_release_pandas}};
        \node[vertex] (printf)
            at  (5,0) {\lstinline{printf}};
        \node[vertex] (pandasaresaved)
            at (0,-10) {\lstinline{pandas_are_saved}};
        \node[vertex] (releasepanda)
            at (10, -5) {\lstinline{release_panda}};
        \node[vertex] (malloc)
            at (10, -10) {\lstinline{malloc}};
        %edges
        \draw[edge, -triangle 90, line width=0.5mm] (main) to (savethepandas);
        \draw[edge, -triangle 90, line width=0.5mm] (main) to (printf);
        \draw[edge, -triangle 90, line width=0.5mm] (savethepandas) to (stopdeforestation);
        \draw[edge, -triangle 90, line width=0.5mm] (savethepandas) to (breedandreleasepandas);
        \draw[edge, -triangle 90, line width=0.5mm] (savethepandas) to (pandasaresaved);
        \draw[edge, -triangle 90, line width=0.5mm] (savethepandas) to (printf);
        \draw[edge, -triangle 90, line width=0.5mm] (breedandreleasepandas) to (releasepanda);
        \draw[edge, -triangle 90, line width=0.5mm] (breedandreleasepandas) to (malloc);
    \end{tikzpicture}
    \caption{Example Call Graph.  The source code associated with this call graph is shown in Figure~\ref{lst:pandasource}}
    \label{fig:pandacallgraph}
\end{figure}
