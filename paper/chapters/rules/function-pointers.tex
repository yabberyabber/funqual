\section{Function Pointers and Indirect Type}\label{sec:rules:funptrs}

Traversing a program for function calls and adding them to the call graph is relatively straightforward.  Knowing exactly what function is being called at the time of parsing makes this process trivial.  This does not account for all function calls, however.  There are multiple cases in modern C++ where a function call is either happening behind the scenes or where the exact callee is not knowable.  This section examines function pointers and explains how they are represented in the call graph.  

As a concrete example, refer to Listing \ref{lst:rules:ptreg}.  In this example, it is literally impossible to know which function \lstinline{strat} is going to point to.  This is a pointed example, but \lstinline{rand} can represent any expression whose result is unknowable during static analysis.  Additionally, in this example there are very clearly only three functions that \lstinline{strat} could point to.  In a real program, there might be thousands of functions and they might not all be listed in one place.  

\noindent\begin{minipage}[t]{\linewidth}
\begin{lstlisting}[language=C,caption={In this example C program, it is impossible to know statically what the value of \lstinline{strat} is.  Because of this, funqual requires the programmer to annotate function pointers with additional type information. },label={lst:rules:ptreg}]
int breed_and_release_pandas() {
    Panda *baby_panda = malloc(sizeof(Panda));
    return release_panda(baby_panda);
}

int (*)() get_random_strategy() {
    switch (rand() % 3) {
        case 0:
            return breed_and_release_pandas;
            break;
        case 1:
            return stop_deforestation;
            break;
        case 2:
            return stop_hunting;
            break;
    }
}

int save_the_pandas() {
    while (!pandas_are_saved()) {
        int (*strat)() = get_random_strategy()
        strat();
    }
    return 0;
}

int main(void) {
    return save_the_pandas();
}
\end{lstlisting}
\end{minipage}

\begin{sloppypar}
If we still intend to use funqual to enforce this $restrict\_indirect\_call(\allowbreak static\_memory, dynamic\_memory)$ rule then we are going to need some additional tools.  Since keeping track of all the possible values of \lstinline{strat} is impractical, we will instead keep track of the type of \lstinline{strat} with respect to this call graph.  Recall that the type of \lstinline{save_the_pandas} is \lstinline{static_memory} and that the type of \lstinline{malloc} is \lstinline{dynamic_memory}.  If we had a function pointer pointing to \lstinline{malloc}, then the type of that function pointer would have to also be \lstinline{dynamic_memory}.  In this example we have a function pointer pointing to \lstinline{breed_and_release_pandas}.  We will say that \lstinline{breed_and_release_pandas} has \textit{indirect type} \lstinline{dynamic_memory} because it calls \lstline{malloc} and so any function pointer that points to \lstinline{breed_and_release_pandas} must have indirect type \lstinline{dynamic_memory}.
\end{sloppypar}

For this reason, when we use function pointers we will have two kinds of type qualifiers:  \textit{direct type} qualifiers and \textit{indirect type} qualifiers.  Direct type refers to the funqual type qualifiers we have explicitly assigned to the pointee.  Indirect type refers to the funqual type qualifiers of all the functions reachable from the pointee.  Direct type for both functions and function pointers must be explicitly annotated in the code.  Indirect types for function pointers must be annotated explicitly but indirect types for functions can be inferred. 

Listing \ref{lst:rules:pandaptrannote} shows the same code as Listing \ref{lst:rules:ptreg} but with function types annotated.  Figure \ref{fig:rules:pandaptrannote} shows the call graph for Listing \ref{lst:rules:pandaptrannote} with the function pointer represented as a cloud.  Notice that we do not need to write any explicit annotations for the indirect type of \lstinline{breed_and_release_pandas}.  Funqual has all the information it needs to statically infer the indirect type of functions.  In this case, the indirect type is \lstinline{dynamic_memory} because \lstinline{breed_and_release_pandas} calls \lstinline{malloc}.  Also notice that \lstinline{strat} has indirect type \lstinline{dynamic_memory}.  This matters because it is \textit{possible} that calling \lstinline{strat} might result in a \lstinline{dynamic_memory} function getting called.  

\noindent\begin{minipage}[t]{\linewidth}
\begin{lstlisting}[language=C,caption={Same example program as Listing \ref{lst:rules:ptreg} but with function pointer type annotations inserted},label={lst:rules:pandaptrannote}]
int breed_and_release_pandas() {
    Panda *baby_panda = malloc(sizeof(Panda));
    return release_panda(baby_panda);
}

int indirect_dynamic_memory (*)() get_random_strategy() {
    switch (rand() % 3) {
        case 0:
            return breed_and_release_pandas;
            break;
        case 1:
            return stop_deforestation;
            break;
        case 2:
            return stop_hunting;
            break;
    }
}

int save_the_pandas() static_memory {
    while (!pandas_are_saved()) {
        int indirect_dynamic_memory (*strat)() =
            get_random_strategy()
        strat();
    }
    return 0;
}

int main(void) {
    return save_the_pandas();
}
\end{lstlisting}
\end{minipage}

\begin{figure}
    \centering
    \begin{tikzpicture}[scale=1.0]
        \tikzset{vertex/.style = {shape=circle,draw,minimum size=8em}}
        \tikzset{edge/.style = {->,> = latex'}}
        % vertices
        \node[vertex] (main) at  (0, 0) {\lstinline{main}};
        \node[vertex, fill=green!50]
            (savethepandas) at  (5, 0) {\lstinline{save_the_pandas}};
        \node[vertex] (pandasaresaved) at (10, 0) {\lstinline{pandas_are_saved}};
        \node[vertex, shape=cloud, pattern=horizontal lines, pattern color=red!20]
            (strat) at (5, -4) {\lstinline{strat}};
        \node[vertex] (stopdeforestation) at  (5, -8) {\lstinline{stop_deforestation}};
        \node[vertex] (stophunting) at (10, -6) {\lstinline{stop_hunting}};
        \node[vertex, pattern=horizontal lines, pattern color=red!20]
            (breedandreleasepandas) at (0, -6)
            {\lstinline{breed_and_release_pandas}};
        \node[vertex] (releasepanda) at (0, -12) {\lstinline{release_panda}};
        \node[vertex, fill=red!50] (malloc) at (5, -12) {\lstinline{malloc}};
        %edges
        \draw[edge, -triangle 90, line width=0.5mm] (main) to (savethepandas);
        \draw[edge, -triangle 90, line width=0.5mm] (savethepandas) to (strat);
        \draw[edge, -triangle 90, line width=0.5mm] (savethepandas) to (pandasaresaved);
        \draw[edge, -triangle 90, line width=0.5mm] (breedandreleasepandas) to (releasepanda);
        \draw[edge, -triangle 90, line width=0.5mm] (breedandreleasepandas) to (malloc);
    \end{tikzpicture}
    \caption[Color-coded Call Graph for Listing \ref{lst:rules:pandaptrannote}.  Functions tagged\break \lstinline{static_memory} are highlighted green and functions tagged\break \lstinline{dynamic_memory} are highlighted red.  Indirect types are represented as horizontal line patterns on a node.  Clouds represent function pointers.]{Color-coded Call Graph for Listing \ref{lst:rules:pandaptrannote}.  Functions tagged \lstinline{static_memory} are highlighted green and functions tagged \lstinline{dynamic_memory} are highlighted red.  Indirect types are represented as horizontal line patterns on a node.  Clouds represent function pointers.}
    \label{fig:rules:pandaptrannote}
\end{figure}

Thanks to the graph based representation of this program, it is clear to see where the error is.  \lstinline{save_the_pandas} calls \lstinline{strat} and it is possible that a call to \lstinline{strat} could result in a call to \lstinline{malloc}.  The indirect type of \lstinline{strat} (notated in Figure \ref{fig:rules:pandaptrannote} as red horizontal lines) is how we keep track of this possibility.  

\subsection{Rules of Assignment}

To properly enforce call graph constraints, funqual checks function pointers in two places:  first when the function pointer is assigned, and second when the function pointer is called.  The rules described in this section are crafted specifically to maintain call graph correctness.  For the purpose of this discussion, we will let $L$ stand for some function pointer and we will let $R$ stand for some function value (the names $L$ and $R$ are a reference to the \lstinline{lvalue} and \lstinline{rvalue} in a typical assignment statement).  

When assigning a function pointer $L$ to point to a function $R$, there are two rules that funqual checks:  the direct type of $L$ must match exactly the direct type of $R$, and the indirect type of $L$ must be a superset of the indirect type of $R$.  For function pointers, both the direct and indirect types must be explicitly annotated in code.  For functions, only the direct type must be explicitly annotated as the indirect type can be inferred.  

These rules are necessary to maintain the soundness of the system.  In order to correctly enforce $require\_direct\_call(X, Y)$, the direct type of $L$ must be contained in $R$ --- otherwise a call to $L$ might be considered valid even if $R$ does not have $Y$ in its type.  In order to correctly enforce $restrict\_direct\_call(X, Y)$, the direct type of $R$ must be contained in $L$ --- otherwise a call to $L$ might be considered valid even if $R$ does not have $Y$ in its type.  Combining both of these requirements means that the direct types of $L$ and $R$ must match exactly.  Lastly, in order to properly enforce $restrict\_indirect\_call(X, Y)$, we need to know all the funqual types that are possibly reachable by calling $L$.  

Table \ref{fig:rules:assignment_table} shows a few examples of valid and invalid assignments.  

\begin{table}
    \caption{Examples of valid and invalid assignments in funqual.  The left two columns show the direct and indirect type of the lvalue respectively.  The next two columns show the direct and indirect type of the rvalue respectively.  The rightmost column shows whether or not that assignment is valid.}
    \centering
    \begin{tabular}{|c|c|c|c|c|}
        \hline
        lvalue & lvalue & rvalue & rvalue & Valid? \\
        direct & indirect & direct & indirect & \\
        \hline
        \hline
        \rowcolor{tablegreen}
        (none) & (none) & (none) & (none) & Valid \\
        \hline
        \rowcolor{tablered}
        static\_memory & (none) & (none) & (none) & Not Valid \\
        \hline
        \rowcolor{tablered}
        (none) & (none) & static\_memory & (none) & Not Valid \\
        \hline
        \rowcolor{tablegreen}
        static\_memory & (none) & static\_memory & (none) & Valid \\
        \hline
        \rowcolor{tablegreen}
        static\_memory & blocking & static\_memory & (none) & Valid \\
        \hline
        \rowcolor{tablered}
        static\_memory & (none) & static\_memory & blocking & Not Valid \\
        \hline
        \rowcolor{tablegreen}
        static\_memory & blocking & static\_memory & blocking & Valid \\
        \hline
        \rowcolor{tablered}
        static\_memory & blocking & static\_memory & nonblocking & Not Valid \\
        \hline
        \rowcolor{tablegreen}
        static\_memory & blocking & static\_memory & nonblocking & Valid \\
        \rowcolor{tablegreen}
         & nonblocking &  & & \\
        \hline
        \rowcolor{tablegreen}
        (none) & blocking & (none) & (none) & Valid \\
        \rowcolor{tablegreen}
         & static\_memory & & & \\ 
        \rowcolor{tablegreen}
         & nonblocking & & & \\ 
        \hline
    \end{tabular}
    \label{fig:rules:assignment_table}
\end{table}


