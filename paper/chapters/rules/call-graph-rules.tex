\section{Call Graph Rules}\label{sec:rules:rules}

% X and Y are type qualifiers (sets of functions)
% E is set of edges
% P is set of paths
% A and B are functions

Each subsection here describes one of the call graph rules supported by funqual.  For each rule, we explain the meaning, provide an algorithm that could enforce it, and present an argument for the algorithm's correctness with respect to the rest of the type system.  The algorithms presented here only return \lstinline{true} or \lstinline{false} depending on whether the graph in question is valid.  The algorithms actually implemented in funqual are slightly more complicated because they print helpful diagnostic messages to the user.  Both sets of algorithms enforce the same rules, though.  

\subsection{Restrict Direct Call}

\begin{center}
    $restrict\_direct\_call(X, Y)$
\end{center}

A restrict direct call rule creates a constraint that disallows functions with direct type $X$ from calling functions with direct type $Y$.  This constraint is relatively permissive because it still allows indirect calls from functions with direct type $X$ to functions with direct type $Y$ but is nonetheless checkable by this type system.

Figure \ref{lst:rules:rules:restrict_direct_call} shows pseudocode for an algorithm that can check a call graph for violations of this rule.  Assume that \lstinline{edges} is a list of objects representing all the calls in the call graph.  

\begin{figure}
    \begin{lstlisting}[gobble=8]
        function enforce_restrict_direct_call(X, Y, edges):
            for edge in edges:
                callee = edge.to
                caller = edge.from

                if X in caller.direct_type and Y in callee.direct_type:
                    return false
            return true
    \end{lstlisting}
    \caption{Pseudocode for an algorithm that can check a $restrict\_direct\_call$ constraint.  This algorithm returns \lstinline{true} if the call graph respects the constraint and \lstinline{false} if the call graph violates it.}
    \label{lst:rules:rules:restrict_direct_call}
\end{figure}

This algorithm runs once per rule and terminates in linear time with respect to the number of edges in the call graph.  To assert the correctness of this algorithm we will categorize each function call in this graph as one of two possibilities:  a call to a standard function, or a call to a function pointer.

\interfootnotelinepenalty=10000

In the case of a standard function call, the correctness is trivial.  The user must have annotated the direct type of both the caller and the callee\footnote{Funqual will check whatever was declared by the programmer --- whether the programmer declared their intent correctly is outside the scope of this research.}.  If a function with direct type $X$ calls a function with direct type $Y$, then \lstinline{edges} will contain such an edge and in checking each edge we will detect it.  

In the case of the function pointer call, we need to also examine all possible assignments of that function pointer.  It is of course possible that the function pointer is null at runtime, but we will consider this type of error to be out of the scope of funqual.  For the sake of this argument, let $P$ stand for any function pointer and $F$ stand for any function.  For an assignment of $F$ into $P$ to be valid, $F$ and $P$ must have the same direct type.  If they do not have the same direct type, then funqual will inform the user of an assignment type violation.  If they do have the same direct type, then \lstinline{edges} will contain an edge into $P$ wherever $P$ is called and that edge will be checked in the same way as a standard function call.  

\subsection{Restrict Indirect Call}

\begin{center}
    $restrict\_indirect\_call(X, Y)$
\end{center}

A restrict indirect call rule creates a constraint that functions with direct type $X$ cannot call functions with direct or indirect type $Y$.  This has the effect of restricting functions with direct type $X$ from calling functions with direct type $Y$, whether that call is direct or indirect.  The need to enforce indirect calls in the presence of function pointers requires us to examine the indirect type of the callee for each edge.  

\begin{figure}
    \begin{lstlisting}[gobble=8]
        function enforce_restrict_indirect_call(X, Y, edges):
            for edge in edges:
                callee = edge.to
                caller = edge.from

                if X in caller.direct_type and Y in callee.indirect_type:
                    return false
            return true
    \end{lstlisting}
    \caption{Pseudocode for an algorithm that can check a $restrict\_indirect\_call$ constraint.  This algorithm returns \lstinline{true} if the call graph respects the constraint and \lstinline{false} if the call graph violates it.}
    \label{lst:rules:rules:restrict_indirect_call}
\end{figure}

Figure \ref{lst:rules:rules:restrict_indirect_call} shows pseudocode for an algorithm that can check a call graph for violations of this rule.  Assume that \lstinline{edges} is a list of objects representing all the calls in the call graph.  

In order to simplify this algorithm, we will assume for the time being that indirect function types are inferred correctly.  For an explanation of the indirect type inference algorithm and for an argument for its correctness, refer to Subsection \ref{sec:rules:rules:inference}.  To assert the correctness of \lstinline{enforce_restrict_indirect_call}, we will again consider each function call in the graph as a member of one of two categories: a call to a standard function, or an invocation of a function pointer.

In the case of a standard function call, the correctness is trivial.  Assume function $A$ with direct type $X$ calls function $B$ with indirect type $Y$.  Since $A$ directly calls $B$, we know that there will be an edge from $A$ to $B$ in the \lstinline{edges} and when the algorithm visits it, the algorithm will terminate with the claim that there is a violation.

In the case of a function pointer invocation, the rules of function pointer assignment come into play.  If, via an invocation of $B$, a function of type $Y$ could eventually be called, then the function pointer must necessarily have $Y$ in its indirect type otherwise there would be an assignment error (for an in-depth argument of this refer to Subsection \ref{sec:rules:rules:inference}).  As a result, when visiting the edge from $A$ to $B$ (where $A$ is the function invoking function pointer $B$), the algorithm will detect that $B$ has indirect type $Y$ and will terminate with the claim that there is a violation.  

\subsection{Require Direct Call}

\begin{center}
    $require\_direct\_call(X, Y)$
\end{center}

A require direct call rule creates a constraint that functions with direct type $X$ can only call functions with direct type $Y$. Much like the restrict direct call rule, this rule is relatively easy to check and can be checked in time linear with respect to the number of edges in the call graph.

Figure \ref{lst:rules:rules:require_direct_call} shows pseudocode for an algorithm that can check a call graph for violations of this rule.  Assume that \lstinline{edges} is a list of objects representing all the calls in the call graph.  

\begin{figure}
    \begin{lstlisting}[gobble=8]
        function enforce_require_direct_call(X, Y, edges):
            for edge in edges:
                callee = edge.to
                caller = edge.from

                if X in caller.direct_type and Y not in callee.direct_type:
                    return false
            return true
    \end{lstlisting}
    \caption{Pseudocode for an algorithm that can check a $require\_direct\_call$ constraint.  This algorithm returns \lstinline{true} if the call graph respects the constraint and \lstinline{false} if the call graph violates it.}
    \label{lst:rules:rules:require_direct_call}
\end{figure}

To assert the correctness of this algorithm, we will categorize every function call as one of two possibilities: a call to a standard function, or a call to a function pointer.

In the case of a call to a standard function, the correctness is trivial.  The user must have annotated the direct type of both the caller and the callee and we take these annotations to be correct.  If a function with direct type $X$ calls any function, then \lstinline{edges} will contain an edge from the caller to the callee.  Checking the direct types of caller and callee exhaustively for every edge in the graph will eventually find any violations.

In the case of a function pointer call, we need to also examine all the possible assignments to that function pointer.  Thankfully the assignment checker already checked the type safety of every function pointer assignment so we will assume that those are correct.  In this case specifically, we can assume that, if the function which is actually called does not have direct type $Y$, then the function pointer which is called in code will also not have direct type $Y$.  This call creates an edge which will certainly be visited by \lstinline{enforce_restrict_direct_call} and so we can be certain that any function pointer invocation will be correctly checked in this regard.

\subsection{Indirect Type Inference}\label{sec:rules:rules:inference}

While the user does not invoke indirect type inference in the same way that the user invokes the other rules, indirect type inference is still an important part of the type safety of funqual.  This subsection explains indirect type inference and argues for the correctness of the algorithm.  

Figure \ref{lst:rules:rules:infer_indirect_type} shows pseudocode for an algorithm that can infer the indirect function type for any function in the call graph.  For the purpose of this function, we will let \lstinline{function} be the function being checked.  We will let \lstinline{edges} be the list of edges in our graph and we will assume that it contains edges to function pointers where those function pointers are called.  We also assume that \lstinline{callee.indirect_type} is populated for function pointers but that it is an empty set for regular functions.  

\begin{figure}
    \begin{lstlisting}[gobble=8]
        function infer_indirect_type(function, edges):
            indirect_types = empty set
            visited = empty set
            to_visit = empty set
            to_visit.add(function)

            while to_visit is not empty:
                curr = to_visit.pop()
                visited.add(curr)

                indirect_types.add_all(curr.direct_type)
                indirect_types.add_all(curr.indirect_type)

                for edge in edges:
                    callee = edge.to
                    caller = edge.from
                    if caller == curr and callee not in visited:
                        to_visit.add(callee)
            return indirect_types
    \end{lstlisting}
    \caption{Pseudocode for an algorithm to infer the indirect type of a function.}
    \label{lst:rules:rules:infer_indirect_type}
\end{figure}

To assert the correctness of this algorithm imagine a function, $F$, from which evaluation eventually (either directly or indirectly) reaches a function, $C$, with type $Y$.  We propose that because of the rules of this type system, it is necessary that $Y$ is in the type of $F$.  To demonstrate this we will break down the type pipeline into its multiple cases.

The first case is that $F$ calls $C$ (either directly or indirectly) but that none of the calls from $F$ to $C$ are function pointer invocations.  In this case, there will be a path in \lstinline{edges} from $F$ to $C$ and because \lstinline{infer_indirect_type} is a breadth first graph traversal starting at $F$, we know that the algorithm will eventually visit $C$.  When the algorithm does visit $C$, it will grab the direct type of $C$ (which contains $Y$) and add it to the indirect type of $F$.  When the algorithm terminates, it will necessarily contain $Y$.  In other words, if there is a path from $F$ to $C$, the indirect type of $F$ will contain the direct and indirect types of $C$.

The second case is that $F$ invokes a function pointer $P$ from which evaluation eventually results in a call to $C$.  In this case, there may or may not be a path in \lstinline{edges} from $F$ to $C$.  However, there will be a path in \lstinline{edges} from $F$ to $P$ and an assignment of $C$ into $P$.  Recall that for an assignment of $C$ into $P$ to typecheck, the direct types of $C$ and $P$ must match and the indirect type of $P$ must contain the indirect type of $C$.  If $Y$ is in the direct type of $C$, then $Y$ must be in the direct type of $P$.  Also if $Y$ is in the indirect type of $C$, then $Y$ must be in the indirect type of $P$.  Since either the direct type or the indirect type of $P$ must contain $Y$, we can reference case one and claim that because there is a path from $F$ to $P$, and because the type of $P$ contains $Y$, then $Y$ will be in the indirect type of $F$.  

The third case is an inductive step.  Assume that $F$ calls $C$ but indirectly through some arbitrary number of function pointer invocations between.  Let $P_0$ be a function pointer through which a call is made to $C$, let $P_1$ be a function pointer through which a call is made to $P_0$, let $P_n$ be a function pointer through which a call is made to $P_{n-1}$, and let $F$ call $P_n$.  According to the logic in case two, if $Y$ is in the direct type of $C$, then it must be in the direct type of $P_0$ or else the assignment will have failed.  In the same way, if $Y$ is in the direct or indirect type of $P_{n-1}$, then it must be in the direct or indirect type of $P_{n}$.  Inductively, $Y$ must be in the direct or indirect type of $P_n$ and because there is a path in \lstinline{edges} from $F$ to $P_n$, $Y$ must end up in the indirect type of $F$.  Lastly, as in case one, any of these calls (either from $F$ to $P_n$, from $P_n$ to $P_{n-1}$, or from $P_0$ to $C$) can be direct or indirect calls and $Y$ will still be in the indirect type of $F$.

This algorithm terminates even in the presence of cycles because it tracks previously visited vertices in \lstinline{visited} and does not visit them again.  Even though these cyclic edges are not followed, the output is still correct because every vertex is visited once.  Assume that $F$ calls $C$ and that $C$ calls $F$.  The algorithm first visits $F$, then visits $C$, but does not visit $F$ again because $F$ was added to \lstinline{visited} when it was first examined.  When $F$ was added to \lstinline{visited}, its direct and indirect type were added to the return value and the edges out of $F$ were added to \lstinline{to_visit}.  All the necessary information was extracted from $F$ on the first visit so visiting it again is not necessary.
