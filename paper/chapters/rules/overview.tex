\section{Overview}\label{sec:rules:overview}

Before doing a deep dive into the specific rules of funqual, let us look at an example.  Recall the save the pandas example from the Background Section.  It is reproduced in this Section as Listing \ref{lst:rules:pandasource} for convenience.  

\noindent\begin{minipage}[t]{\linewidth}
\begin{lstlisting}[language=C,caption={Example C program.  Running this code in a production environment may not actually save the pandas},label={lst:rules:pandasource}]
int breed_and_release_pandas() {
    Panda *baby_panda = malloc(sizeof(Panda));
    release_panda(baby_panda);
}

int save_the_pandas() {
	stop_deforestation());
	if (pandas_are_saved()) {
		printf("Stopping deforestation saved the pandas!\n");
		return 1;
	}

	breed_and_release_pandas();
	if (pandas_are_saved()) {
		printf("Breeding pandas in captivation and releasing them has saved the pandas!\n");
		return 1;
	}
	return 0;
}

int main(void) {
	if (save_the_pandas()) {
		printf("The pandas have been saved!\n");
	}
}
\end{lstlisting}
\end{minipage}

\begin{sloppypar}
Let us now imagine that there is some constraint whereby \lstinline{save_the_pandas} should not allocate memory. As programmers we would like to believe that we are disciplined enough to remember this rule and enforce it ourselves. In practice, self-regulation like this often ends poorly. As a result we would like a tool like funqual to enforce this constraint automatically. To accomplish this we will create two type qualifiers: \lstinline{static_memory} and \mbox{\lstinline{dynamic_memory}.} We will also create one rule: $restrict\_indirect\_call(static\_memory, dynamic\_memory)$. When the programmer qualifies a function with \lstinline{static_memory}, that declares the intent that this function will \textit{never} allocate memory on the heap. When the programmer qualifies a function with \lstinline{dynamic_memory}, that declares the intent that this function always allocates memory on the heap. The rule $restrict\_indirect\_call(static\_memory, dynamic\_memory)$ tells funqual that \lstinline{static_memory} functions are not allowed to call \lstinline{dynamic_memory} functions either directly or indirectly. If it is possible for a \lstinline{static_memory} function to reach a \lstinline{dynamic_memory} function, then the rule has been violated and funqual should inform the user. 
\end{sloppypar}

In the example about saving the pandas, we would qualify \lstinline{save_the_pandas} as \lstinline{static_memory} and we would qualify \lstinline{malloc} as \lstinline{dynamic_memory}.  Figure \ref{fig:coloredpandacallgraph} shows the call graph for Listing \ref{lst:rules:pandasource} with \lstinline{static_memory} functions marked green and with \lstinline{dynamic_memory} functions marked red.

\begin{figure}
    \centering
    \begin{tikzpicture}[scale=1.0]
        \tikzset{vertex/.style = {shape=circle,draw,minimum size=8em}}
        \tikzset{edge/.style = {->,> = latex'}}
        % vertices
        \node[vertex] (main) at  (0,0) {$main$};
        \node[vertex, fill=green!50] (savethepandas) at  (0,-5) {$save\_the\_pandas$};
        \node[vertex] (stopdeforestation) at  (5,-5) {$stop\_deforestation$};
        \node[vertex] (breedandreleasepandas) at  (5,-10) {$breed\_and\_release\_pandas$};
        \node[vertex] (printf) at  (5,0) {$printf$};
        \node[vertex] (pandasaresaved) at (0,-10) {$pandas\_are\_saved$};
        \node[vertex] (releasepanda) at (10, -5) {$release\_panda$};
        \node[vertex, fill=red!50] (malloc) at (10, -10) {$malloc$};
        %edges
        \draw[edge, -triangle 90, line width=0.5mm] (main) to (savethepandas);
        \draw[edge, -triangle 90, line width=0.5mm] (main) to (printf);
        \draw[edge, -triangle 90, line width=0.5mm] (savethepandas) to (stopdeforestation);
        \draw[edge, -triangle 90, line width=0.5mm] (savethepandas) to (breedandreleasepandas);
        \draw[edge, -triangle 90, line width=0.5mm] (savethepandas) to (pandasaresaved);
        \draw[edge, -triangle 90, line width=0.5mm] (savethepandas) to (printf);
        \draw[edge, -triangle 90, line width=0.5mm] (breedandreleasepandas) to (releasepanda);
        \draw[edge, -triangle 90, line width=0.5mm] (breedandreleasepandas) to (malloc);
    \end{tikzpicture}
    \caption{Color-coded Call Graph for Listing \ref{lst:pandasource}.  Functions tagged \lstinline{static_memory} are highlighted green and functions tagged \lstinline{dynamic_memory} are highlighted red.}
    \label{fig:coloredpandacallgraph}
\end{figure}


\begin{sloppypar}
By turning the program into a directed graph and by assigning types to the vertices, we have transformed the problem of type safety into a graph problem.  A question like \textit{are there any static\_memory functions that inadvertently call dynamic\_memory functions} essentially boils down to \textit{are there any paths from green vertices to red vertices}.  In this example, the answer to that question is yes.  In the code, \lstinline{save_the_pandas} calls \lstinline{breed_and_release_pandas} which calls \lstinline{malloc} constituting an illicit call.  Equivalently, \lstinline{save_the_pandas} has an edge to \lstinline{breed_and_release_pandas} which has an edge to \lstinline{malloc} constituting an illicit path.  A well-typed program has no paths from green vertices to red vertices.  A poorly-typed program will have at least one path.  
\end{sloppypar}


