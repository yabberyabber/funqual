\section{Restricting API available during kernel initialization}\label{sec:app:pre-malloc}

Kernel development is complicated for a variety of reasons.  One reason that makes it particularly complicated is that not all of the standard libraries are available from certain contexts in the kernel.  Just like in section \ref{sec:app:reentrancy}, it would be a serious issue if we were to call \lstinline{malloc} or \lstinline{printf} from within an interrupt handler.  In addition to that, we also need to ensure that these functions are not called before their associated interfaces are initialized.  As an example, the use of \lstinline{malloc} depends on the page table having been initialized, and so there are contexts within kernel initialization where a call to \lstinline{malloc} would be inappropriate.  We would like to prevent this error as well as errors like it. 

To demonstrate how funqual can be applied to this problem, we take an assignment for Operating Systems II (CSC454) where students develop their own simple x86\_64 kernel, and augment it with function type qualifiers and a rules file.  The kernel consists of several subsystems that are each initialized in sequence.  These subsystems, in order, are: a VGA subsystem to display text on the screen, a PC2 subsystem to poll the keyboard and mouse for input, a subsystem to schedule interrupts and register interrupt handlers, an interface to send and receive text over the system's serial interface, a memory manager to allocate physical pages and add them to the page table, a scheduler to run multiple processes, and lastly a set of processes which each draw a snake crawling around the screen like in section \ref{sec:app:reentrancy}.  These subsystems all depend on things in the previous subsystems.  The keyboard subsystem depends on the VGA subsystem to display the keys the user pressed, the physical memory manager depends on the interrupt subsystem to listen for page faults, \lstinline{malloc} depends on the memory manager, and so on.  It is very easy in the early stages of boot-up to accidentally call a function in a subsystem that has not been initialized.  Sometimes these calls are hidden by a few edges in the call graph, making it hard for a human to detect them.  

To solve this problem, we use several type qualifiers:  \lstinline{pre_vga}, \lstinline{vga}, \lstinline{pre_pc2}, \lstinline{pc2}, \lstinline{pre_irq}, \lstinline{irq}, \lstinline{pre_ser}, \lstinline{ser}, \lstinline{pre_mmu}, \lstinline{mmu}, \lstinline{pre_proc}, and \lstinline{proc}.  We also create rules for each subsystem that prevent functions tagged \lstinline{pre_XXX} from calling functions in any of the subsystems that involve \lstinline{XXX}.  Listing \ref{lst:app:pre-malloc:rules} shows the rules file used to support this.  

The subsystems are initialized in the following order: VGA, PC2, Interrupt Request (IRQ), Serial, Memory Management Unit (MMU), and process manager (proc).  For each subsystem, there is an \lstinline{_init} function which is called to set the subsystem up.  The \lstinline{_init} function for each subsystem is annotated with \lstinline{pre_XXX} for each subsystem that it proceeds.  Listing \ref{lst:app:pre-malloc:annote} shows the annotations that were added to accomplish this.

\noindent\begin{minipage}[t]{\linewidth}
\begin{lstlisting}[caption={Rules file for a simple kernel written for CSC454.  The rules written here are intended to prevent code which runs before a subsystem is initialized from calling any function that depend on the subsystem.},label={lst:app:pre-malloc:rules}]
rule restrict_indirect_call pre_vga vga
rule restrict_indirect_call pre_vga pc2
rule restrict_indirect_call pre_vga irq
rule restrict_indirect_call pre_vga ser
rule restrict_indirect_call pre_vga mmu
rule restrict_indirect_call pre_vga proc

rule restrict_indirect_call pre_pc2 pc2
rule restrict_indirect_call pre_pc2 irq
rule restrict_indirect_call pre_pc2 ser
rule restrict_indirect_call pre_pc2 mmu
rule restrict_indirect_call pre_pc2 proc

rule restrict_indirect_call pre_irq irq
rule restrict_indirect_call pre_irq ser
rule restrict_indirect_call pre_irq mmu
rule restrict_indirect_call pre_irq proc

rule restrict_indirect_call pre_ser ser
rule restrict_indirect_call pre_ser mmu
rule restrict_indirect_call pre_ser proc

rule restrict_indirect_call pre_mmu mmu
rule restrict_indirect_call pre_mmu proc

rule restrict_indirect_call pre_proc proc
\end{lstlisting}
\end{minipage}

% separator between 2 listings

\noindent\begin{minipage}[t]{\linewidth}
\begin{lstlisting}[language=c++,caption={Lines inserted into C source for a simple kernel in order to prevent subsystems from depending on interfaces not yet initialized.},label={lst:app:pre-malloc:annote}]
// for VGA subsystem
bool VGA_init() QTAG(pre_pc2);
void set_char_at(int row, int col, char character, char attributes)
    QTAG(vga) QTAG(pre_pc2);

// for PC2 subsystem
void PC2_init() QTAG(pre_irq);
bool get_char(char *ret) QTAG(pc2) QTAG(pre_irq);

// for interrupt request subsystem
void IRQ_init() QTAG(pre_ser);
void IRQ_set_handler(int irq, irq_handler_t handler, void *args)
    QTAG(irq) QTAG(pre_ser);

// for serial subsystem
void SER_init() QTAG(pre_mmu);
int SER_write(const char *buff, int len) QTAG(ser) QTAG(pre_mmu);

// for memory management unit subsystem
void MMU_pt_init() QTAG(pre_proc);
void *MMU_alloc_page() QTAG(mmu) QTAG(pre_proc);
void MMU_free_page(void *page) QTAG(mmu) QTAG(pre_proc);

// for multiprocessing subsystem
void PROC_init();
Process_t *PROC_create_kthread(kproc_t entry_point, void *arg)
    QTAG(proc);
void PROC_run() QTAG(proc);
\end{lstlisting}
\end{minipage}

This project contained 6034 lines of C code spread over 36 files with 291 edges in the call graph\footnote{Line count achieved using the \lstinline{cloc} utility not including comments or blank lines.  Many of these lines were machine-generated.}.  Funqual analyzed this source in about 2.2 seconds --- 0.3 seconds were spend in libClang parsing source files, 1.9 seconds were spent building the call graph, less than 0.000,05 seconds were spent checking the call graph, and less than 0.000,05 seconds were spent checking function pointer assignments.  

Funqual did detect errors in this source.  Listing \ref{lst:app:pre-malloc:output} shows the output from running funqual.  According to this output, funqual detected several cases where a \lstinline{pre_ser} function called a \lstinline{ser} function.  Specifically, \lstinline{IRQ_init} called \lstinline{printk} which eventually calls \lstinline{SER_write}.  \lstinline{printk} in this project operates just like \lstinline{printf}.  The only difference is that \lstinline{printk} outputs to both VGA and to serial.  This rule violation represents an actual error that existed in the project which was not detected until funqual found it.  Ideally, \lstinline{SER_write} just fills a buffer; this buffer is emptied when the serial port sends an interrupt requesting data.  However, if the serial interface isn't operating, that buffer might fill up causing the caller to block until space becomes available in the buffer.  If we had filled the entire 1024 character buffer before initializing the serial interface, the kernel would have hanged indefinitely.

\noindent\begin{minipage}[t]{\linewidth}
\begin{lstlisting}[caption={Output from funqual when run on simple OS kernel},label={lst:app:pre-malloc:output}]
Rule violation: `ser` function indirectly called from `pre_ser` context
        Path:   src/IRQ.c::IRQ_init() (12,6)
        -calls: src/io.c::printk(const char *, ...) (7,5)
        -calls: src/io.c::print_char(char) (9,6)
        -calls: src/main.c::SER_write(const char *, int) (7,5)

Rule violation: `ser` function indirectly called from `pre_ser` context
        Path:   src/IRQ.c::IRQ_init() (12,6)
        -calls: src/io.c::printk(const char *, ...) (7,5)
        -calls: src/io.c::print_str(const char *) (10,6)
        -calls: src/main.c::SER_write(const char *, int) (7,5)

Rule violation: `ser` function indirectly called from `pre_ser` context
        Path:   src/IRQ.c::IRQ_init() (12,6)
        -calls: src/io.c::printk(const char *, ...) (7,5)
        -calls: src/io.c::print_long_hex(uint64_t, bool) (13,6)
        -calls: src/io.c::print_str(const char *) (10,6)
        -calls: src/main.c::SER_write(const char *, int) (7,5)

Rule violation: `ser` function indirectly called from `pre_ser` context
        Path:   src/IRQ.c::IRQ_init() (12,6)
        -calls: src/io.c::printk(const char *, ...) (7,5)
        -calls: src/io.c::print_decimal(int64_t) (14,6)
        -calls: src/io.c::print_str(const char *) (10,6)
        -calls: src/main.c::SER_write(const char *, int) (7,5)
\end{lstlisting}
\end{minipage}

Observing the version history of this codebase tells the full story of how this bug occurred.  Initially, \lstinline{printk} only printed to VGA.  At that point in time, it was okay to call \lstinline{printk} from \lstinline{IRQ_init} because \lstinline{printk} only depended on VGA.  However, at some point a decision was made that \lstinline{printk} should print to both VGA and serial.  At this point, a simple modification was made to \lstinline{print_str} without considering all the ways that \lstinline{printk} was used. 

This sort of mistake is probably very familiar to many programmers and it might have existed for quite a while before appearing.  Imagine what the symptoms would have been: the programmer would have added an extra \lstinline{printk} to \lstinline{IRQ_init} and the programmer would have observed that the program hanged.  Why did it hang, though?  In kernel development, there are a lot of things that could cause a hang, but a simple call to \lstinline{printk} might not be the first thing one would check.  It would take quite a lot of digging to go from \lstinline{printk} to \lstinline{print_char} to \lstinline{SER_write} then to see that the issue was there.  

This situation really demonstrates the strength of funqual.  The programmer cannot possibly be cognisant of every location from which a function is called.  What's more, the programmer is usually unaware of paths between functions, especially when there are several levels of indirection in-between them.  Letting the programmer express their intuition as hard-coded rules and using a tool to check those rules automatically enables the programmer to be confident that errors like this don't occur in practice.  
