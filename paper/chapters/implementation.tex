\chapter{Implementation}\label{sec:implementation}

This section contains information about the funqual tool including a discussion of how to use it, how it works, and what its limitations are. 

\section{Operation}\label{sec:operation}

Funqual is a tool that takes in C++ source code and a set of call graph rules and outputs a list of rule violations, if any exist.  Section \ref{sec:operate:annote} demonstrates how to annotate C++ source code with funqual type qualifiers.  Section \ref{sec:operate:rules} explains the syntax for writing down rules in the rules file.  Section \ref{sec:operate:run} shows the syntax for running funqual from the command line.  Finally, Section \ref{sec:operate:output} contains a few examples of programs, rules files, and the output from funqual.  

\subsection{Function Qualifier Annotations with QTAG and QTAG\_IND}\label{sec:operate:annote}

One of the goals of funqual was that it be entirely compatible with the C++17 standard.  As such, funqual does not add any syntax to the language that would prevent annotated programs from being used by other tools (such as gcc or cppchecker).  Additionally, any C++17 code that exists ``in the wild'' should be compatible with funqual with no modification.  To this end, we use the existing C++17 annotation syntax to insert funqual type qualifiers.  

For clarity and convenience we assume the following macros are in scope.  These macros abbreviate the syntax for inserting the direct and indirect type qualifiers into program source.  In practice, this macro can be inserted into the code alongside the annotations or can be placed in a utility library:

\noindent\begin{minipage}[t]{\linewidth}
\begin{lstlisting}[language=C,caption={}]
#ifndef FUNQUAL
#define FUNQUAL
// For direct type:
#define QTAG(TAG) \
    __attribute__((annotate("funqual::" #TAG)))
// For indirect type:
#define QTAG_IND(TAG) \
    __attribute__((annotate("funqual_indirect::" #TAG)))
#endif
\end{lstlisting}
\end{minipage}

Note that the \lstilnine{\_\_attribute\_\_((annotate(foobar)))} syntax is generally used for compiler-specific directives (like packed, align(8), noreturn, etc) and that attributes unknown by the compiler are simply ignored.  This allows us to insert information into the AST that is available after parsing but which will not affect compilation.

Below is an example of the syntax for adding type qualifiers to a function.  The function below has two qualified types: \lstinline{static_memory} and \lstinline{no_io}.

\noindent\begin{minipage}[t]{\linewidth}
\begin{lstlisting}[language=C]
int main() QTAG(static_memory) QTAG(no_io) {
    return 0;
}
\end{lstlisting}
\end{minipage}

Below is an example of the syntax for adding type qualifiers to a method prototype within a class.  The function below has qualified type \lstinline{static_memory}.

\noindent\begin{minipage}[t]{\linewidth}
\begin{lstlisting}[language=C++]
class Panda {
    Panda() QTAG(static_memory);
};
\end{lstlisting}
\end{minipage}

Below is an example of the syntax for adding a type qualifier to a function pointer.  The function pointer below has qualified type \lstinline{static_memory}.

\noindent\begin{minipage}[t]{\linewidth}
\begin{lstlisting}[language=C++]
int QTAG(static_memory) (*func)(int, int);
\end{lstlisting}
\end{minipage}

Functions in the standard library can be annotated by simply repeating their prototype and adding a type qualifier annotation.  During the first phase of type checking, funqual will scrape the entire codebase and determine the union of all type annotations for each function symbol.  In the example below, \lstinline{malloc} has two type qualifiers:  \lstinline{dynamic_memory} and \lstinline{blocking}.  Lines 1 and 3 could appear in the same file or in different files.  There is no limit to the number of type qualifiers that can be applied to a function.

\noindent\begin{minipage}[t]{\linewidth}
\begin{lstlisting}[language=C]
void *malloc(size_t size) QTAG(dynamic_memory);

void *malloc(size_t size) QTAG(blocking);
\end{lstlisting}
\end{minipage}

Function pointers must also be annotated with their indirect type.  For a primer on the rules regarding indirect type and function pointer assignment, refer to Section \ref{sec:rules:funptrs}.  Below is an example of a function pointer with the indirect type \lstinline{blocking}.

\noindent\begin{minipage}[t]{\linewidth}
\begin{lstlisting}[language=C++]
int QTAG_IND(blocking) (*func)(int, int);
\end{lstlisting}
\end{minipage}

\subsection{Constrain the World!  Writing a Rules File}\label{sec:operate:rules}

Call graph rules are inserted into special files called rules files.  By convention, rules files have the file extension \lstinline{.qtag} but this convention is optional.  Below is an example of a rules file that shows a few examples of each rule type:

\noindent\begin{minipage}[t]{\linewidth}
\begin{lstlisting}
rule restrict_indirect_call static_memory dynamic_memory
rule restrict_direct_call nonblocking blocking
rule require_direct_call nonblocking nonblocking
\end{lstlisting}
\end{minipage}

\begin{sloppypar}
This rules file contains three rules: $restrict\_indirect\_call(\allowbreak static\_memory,\allowbreak dynamic\_memory)$, $restrict\_direct\_call(\allowbreak nonblocking,\allowbreak blocking)$, and $require\_direct\_call(\allowbreak nonblocking,\allowbreak nonblocking)$.  As shown in this file, there is no process of declaring a type qualifier.  They are brought into existence simply by referencing them.  
\end{sloppypar}

In addition to specifying rules in a rules file, funqual also allows the user to specify additional function qualifiers in this file.  In order to do this, the user must determine the clang Unified Symbol Resolution for the given symbol.  This is a string that uniquely identifies the symbol across all translation units - it contains more information than the fully qualified name of the symbol because it needs to differentiate between static symbols in different translation units and it needs to differentiate between overloaded identifiers within the same translation unit.  The Listing below demonstrates the syntax for adding the \lstinline{dynamic_memory} qualifier to the stdlib \lstinline{malloc}:

\noindent\begin{minipage}[t]{\linewidth}
\begin{lstlisting}
tag c:@F@malloc dynamic_memory
\end{lstlisting}
\end{minipage}

\subsection{Running Funqual}\label{sec:operate:run}

Funqual can be run from the command line.  There are two kinds of arguments:  translation units and rules files.  Arguments preceded by \lstinline{-t} or \lstinline{--tags-file} will be interpreted as a rules file.  All other arguments will be interpreted as translation units.  Funqual needs to be passed every translation unit in a project in order for it to create a representative call graph for the codebase.  Below is an example command for running funqual.  This command will pass in every \lstinline{.cpp} file in the current directory and any subdirectories and will also pass in a rules file called \lstinline{rules.qtag} in the current directory.

\noindent\begin{minipage}[t]{\linewidth}
\begin{lstlisting}
funqual *.cpp -t rules.qtag
\end{lstlisting}
\end{minipage}

\subsection{Example Output}\label{sec:operate:output}

Below is the output of running funqual on Listing \ref{lst:rules:inheritance}.  Not only does funqual detect the presence of a rule violation, it also shows the exact sequence of calls that represent the violation.  This information helps the user know that their code contains a type error and also helps the user to correct the error.

\noindent\begin{minipage}[t]{\linewidth}
\begin{lstlisting}
Rule violation: `dynamic_memory` function indirectly called from `static_memory` context
        Path:   main.cpp::main() (38,5)
        -calls: main.cpp::RedPanda::Feed(int) (31,18)
        -calls: main.cpp::(#include)::malloc(size_t) (466,14)
\end{lstlisting}
\end{minipage}

Funqual will, by default, output every illegal path in a program.  This has the potential to generate a lot of output for larger programs with many violations.  

\section{Practical Limitations}

Because of complexities in parsing C++, certain applications of function pointers are not currently checkable by funqual.  Specifically, any expression where the lvalue or rvalue in a function pointer assignment is anything other than a raw variable can not be checked.  Listing \ref{sec:imp:funptrfails} shows a few examples of assignment expressions that funqual cannot check correctly.

\noindent\begin{minipage}[t]{\linewidth}
\begin{lstlisting}[language=c++,label={sec:imp:funptrfails},caption={Examples of function pointer assignment expressions that are not checked correctly by funqual}]
void *(**array)(size_t size) QTAG(dynamic_memory);

//assignment not checkable because the array dereference in lvalue
array[0] = malloc;

struct {
    void *(*field)(size_t size);
} structure;

struct structure s;

//assignment not checkable because of field dereference in lvalue
s.field = malloc;

void *(**arr)(size_t);

//assignment not checkable because of array dereference in rvalue
arr = array[0];
\end{lstlisting}
\end{minipage}

The difficulty here arises from determining the type of an expression using the libClang API.  Types containing function pointers can be annotated by inserting \lstinline{QTAG} and \lstinline{QTAG_IND} annotations.  Additionally, the type of any expression in the libClang AST can be queried.  However, the type returned by libClang when querying for the type of an expression will not contain type qualifier annotations.  In order to get the type annotations for lvalues and rvalues in an assignment, funqual needs to look up the declaration of the identifier and parse it.  Types in C++ can be arbitrarily nested and function pointers can be hidden within complicated types.  In order to limit complexity of the funqual tool, a decision was made that funqual would only support assignments to and from raw identifiers.  

For the same reason, funqual does not support casting function types to coerce them into having certain type qualifiers.  When determining the type of the result of a cast expression, libClang ignores the type qualifier annotations and so the cast expression loses that information after parsing.  
