\chapter{Type Rules}\label{sec:rules}

Funqual checks a program against a type system.  It is a tool that takes in source code as well some user-defined call graph rules, does some computation, and prints one of two things: ``This program is well-typed'', or ``This program is not well typed'' (in practice the later case also comes with an explanation as to why the program is not well-typed).  If a program is well-typed, then it is free from call graph rule violations.  If a program is not well-typed, then it may contain one or more errors.

This chapter contains an overview of the rules implemented by funqual as well as a brief exploration of what needs to happen behind the scenes in order to correctly check these rules.  Section \ref{sec:rules:overview} demonstrates the big picture of what these rules are trying to accomplish.  Section \ref{sec:rules:funptrs} explains how type qualifiers are applied to function pointers and how funqual checks them.  Section \ref{sec:rules:rules} is a detailed explanation of each of the call graph constraints supported by funqual.  Finally, Section \ref{sec:rules:special} explains a few special cases and explains how funqual handles them to create a complete call graph.

Note that this chapter focuses only on the conceptual design of funqual and its type system.  For details on how it is implemented or how to use it, refer to \mbox{Chapter \ref{sec:implementation}}.

\section{Overview}\label{sec:rules:overview}

Before doing a deep dive into the specific rules of funqual, let us look at an example.  Recall the save the pandas example from Section \ref{sec:background}.  It is reproduced in this Section as Figure \ref{lst:rules:pandasource} for convenience.  

\begin{figure}
    \begin{lstlisting}[language=C,gobble=8]
        int breed_and_release_pandas() {
            Panda *baby_panda = malloc(sizeof(Panda));
            release_panda(baby_panda);
        }

        int save_the_pandas() {
            stop_deforestation());
            if (pandas_are_saved()) {
                printf("Stopping deforestation saved the pandas!\n");
                return 1;
            }

            breed_and_release_pandas();
            if (pandas_are_saved()) {
                printf("Breeding pandas in captivation and releasing them has saved the pandas!\n");
                return 1;
            }
            return 0;
        }

        int main(void) {
            if (save_the_pandas()) {
                printf("The pandas have been saved!\n");
            }
        }
    \end{lstlisting}
    \caption{Example C program.  Running this code in a production environment may not actually save the pandas}
    \label{lst:rules:pandasource}
\end{figure}

Let us now imagine that there is some constraint whereby \lstinline{save_the_pandas} should not allocate memory. As programmers we would like to believe that we are disciplined enough to remember this rule and enforce it ourselves. In practice, self-regulation like this often ends poorly. As a result we would like a tool like funqual to enforce this constraint automatically.

\begin{sloppypar}
Funqual allows us as programmers to create our own type qualifiers and to apply whatever meaning we want to those qualifiers.  In this particular case we create two type qualifiers: \lstinline{static_memory} and \mbox{\lstinline{dynamic_memory}.} We also create one rule: $restrict\_indirect\_call(static\_memory, dynamic\_memory)$. When the programmer qualifies a function with \lstinline{static_memory}, that declares the intent that this function will \textit{never} allocate memory on the heap. When the programmer qualifies a function with \lstinline{dynamic_memory}, that declares the intent that this function always allocates memory on the heap\footnote{The meanings of these type qualifiers are as determined by the programmer; without a rule to operate on them, funqual will completely ignore the type qualifiers.}. The rule $restrict\_indirect\_call(static\_memory, dynamic\_memory)$ tells funqual that \lstinline{static_memory} functions are not allowed to call \lstinline{dynamic_memory} functions either directly or indirectly. If it is possible for a \lstinline{static_memory} function to reach a \lstinline{dynamic_memory} function, then the rule has been violated and funqual should inform the user.
\end{sloppypar}

In the example about saving the pandas, we would qualify \lstinline{save_the_pandas} as \lstinline{static_memory} and we would qualify \lstinline{malloc} as \lstinline{dynamic_memory}.  Figure \ref{fig:coloredpandacallgraph} shows the call graph for Figure \ref{lst:rules:pandasource} with \lstinline{static_memory} functions marked green and with \lstinline{dynamic_memory} functions marked red.

\begin{figure}
    \centering
    \begin{tikzpicture}[scale=1.0]
        \tikzset{vertex/.style = {shape=circle,draw,minimum size=8em}}
        \tikzset{edge/.style = {->,> = latex'}}
        % vertices
        \node[vertex] (main) at  (0,0) {$main$};
        \node[vertex, fill=green!50] (savethepandas) at  (0,-5) {$save\_the\_pandas$};
        \node[vertex] (stopdeforestation) at  (5,-5) {$stop\_deforestation$};
        \node[vertex] (breedandreleasepandas) at  (5,-10) {$breed\_and\_release\_pandas$};
        \node[vertex] (printf) at  (5,0) {$printf$};
        \node[vertex] (pandasaresaved) at (0,-10) {$pandas\_are\_saved$};
        \node[vertex] (releasepanda) at (10, -5) {$release\_panda$};
        \node[vertex, fill=red!50] (malloc) at (10, -10) {$malloc$};
        %edges
        \draw[edge, -triangle 90, line width=0.5mm] (main) to (savethepandas);
        \draw[edge, -triangle 90, line width=0.5mm] (main) to (printf);
        \draw[edge, -triangle 90, line width=0.5mm] (savethepandas) to (stopdeforestation);
        \draw[edge, -triangle 90, line width=0.5mm] (savethepandas) to (breedandreleasepandas);
        \draw[edge, -triangle 90, line width=0.5mm] (savethepandas) to (pandasaresaved);
        \draw[edge, -triangle 90, line width=0.5mm] (savethepandas) to (printf);
        \draw[edge, -triangle 90, line width=0.5mm] (breedandreleasepandas) to (releasepanda);
        \draw[edge, -triangle 90, line width=0.5mm] (breedandreleasepandas) to (malloc);
    \end{tikzpicture}
    \caption{Color-coded Call Graph for Listing \ref{lst:pandasource}.  Functions tagged \lstinline{static_memory} are highlighted green and functions tagged \lstinline{dynamic_memory} are highlighted red.}
    \label{fig:coloredpandacallgraph}
\end{figure}


\begin{sloppypar}
By turning the program into a directed graph and by assigning types to the vertices, we have transformed the problem of type qualifier rule satisfaction into a graph problem.  A question like \textit{are there any static\_memory functions that inadvertently call dynamic\_memory functions} essentially boils down to \textit{are there any paths from green vertices to red vertices}.  In this example, the answer to that question is yes.  In the code, \lstinline{save_the_pandas} calls \lstinline{breed_and_release_pandas} which calls \lstinline{malloc} constituting an illicit call.  Equivalently, \lstinline{save_the_pandas} has an edge to \lstinline{breed_and_release_pandas} which has an edge to \lstinline{malloc} constituting an illicit path.  A well-typed program has no paths from green vertices to red vertices.  A poorly-typed program will have at least one path.  
\end{sloppypar}




\section{Function Pointers and Indirect Type}\label{sec:rules:funptrs}

Traversing a program for function calls and adding them to the call graph is relatively straightforward.  Knowing exactly what function is being called at the time of parsing makes this process trivial.  This does not account for all function calls, however.  There are multiple cases in modern C++ where a function call is either happening behind the scenes or where the exact callee is not knowable.  This section examines function pointers and explains how they are represented in the call graph.  

As a concrete example, refer to Listing \ref{lst:rules:ptreg}.  In this example, it is literally impossible to know what function \lstinline{strat} is going to point to.  This is a pointed example, but \lstinline{rand} can represent any expression whose result is unknowable during static analysis.  Additionally, in this example there are very clearly only three functions that \lstinline{strat} could point to.  In a real program, there might be thousands of functions and they might not all be listed in one place.  

\noindent\begin{minipage}[t]{\linewidth}
\begin{lstlisting}[language=C,caption={In this example C program, it is impossible to know statically what the value of \lstinline{strat} is.  Because of this, funqual requires the programmer to annotate function pointers with additional type information. },label={lst:rules:ptreg}]
int breed_and_release_pandas() {
    Panda *baby_panda = malloc(sizeof(Panda));
    return release_panda(baby_panda);
}

int (*)() get_random_strategy() {
    switch (rand() % 3) {
        case 0:
            return breed_and_release_pandas;
            break;
        case 1:
            return stop_deforestation;
            break;
        case 2:
            return stop_hunting;
            break;
    }
}

int save_the_pandas() {
    while (!pandas_are_saved()) {
        int (*strat)() = get_random_strategy()
        strat();
    }
    return 0;
}

int main(void) {
    return save_the_pandas();
}
\end{lstlisting}
\end{minipage}

\begin{sloppypar}
If we still intend to use funqual to enforce this $restrict\_indirect\_call(\allowbreak static\_memory, dynamic\_memory)$ rule then we are going to need some additional tools.  Since keeping track of all the possile values of \lstinline{strat} is impractical, we will instead keep track of the type of \lstinline{strat} with respect to this call graph.  Recall that the type of \lstinline{save_the_pandas} is \lstinline{static_memory} and that the type of \lstinline{malloc} is \lstinline{dynamic_memory}.  If we had a function pointer pointing to \lstinline{malloc}, then the type of that function pointer would have to also be \lstinline{dynamic_memory}.  In this example we have a function pointer pointing to \lstinline{breed_and_release_pandas}.  We will say that \lstinline{breed_and_release_pandas} has \textit{indirect type} \lstinline{dynamic_memory} because it calls \lstline{malloc} and so any function pointer that points to \lstinline{breed_and_release_pandas} must have indirect type \lstinline{dynamic_memory}.
\end{sloppypar}

For this reason, when we use function pointers we will have two kinds of type qualifiers:  \textit{direct type} qualifiers and \textit{indirect type} qualifiers.  Direct type refers to the funqual type qualifiers we have explicitly assigned to the pointee.  Indirect type refers to the funqual type qualifiers of all the functions reachable from the pointee.  Direct type for both functions and function pointers must be explicitly annotated in the code.  Indirect types for function pointers must be annotated explicitly but can be inferred automatically for functions. 

Listing \ref{lst:rules:pandaptrannote} shows the same code as Listing \ref{lst:rules:ptreg} but with function types annotated.  Figure \ref{fig:rules:pandaptrannote} shows the call graph for Listing \ref{lst:rules:pandaptrannote} with the function pointer represented as a cloud.  Notice that we do not need to write any explicit annotations for the indirect type of \lstinline{breed_and_release_pandas}.  Funqual has all the information it needs to statically infer the indirect type of functions.  In this case, the indirect type is \lstinline{dynamic_memory} because \lstinline{breed_and_release_pandas} calls \lstinline{malloc}.  Also notice that \lstinline{strat} has indirect type \lstinline{dynamic_memory}.  This matters because it is \textit{possible} that calling \lstinline{strat} might result in a \lstinline{dynamic_memory} function getting called.  

\noindent\begin{minipage}[t]{\linewidth}
\begin{lstlisting}[language=C,caption={Same example program as Listing \ref{lst:rules:ptreg} but with function pointer type annotations inserted},label={lst:rules:pandaptrannote}]
int breed_and_release_pandas() {
    Panda *baby_panda = malloc(sizeof(Panda));
    return release_panda(baby_panda);
}

int indirect_dynamic_memory (*)() get_random_strategy() {
    switch (rand() % 3) {
        case 0:
            return breed_and_release_pandas;
            break;
        case 1:
            return stop_deforestation;
            break;
        case 2:
            return stop_hunting;
            break;
    }
}

int save_the_pandas() static_memory {
    while (!pandas_are_saved()) {
        int indirect_dynamic_memory (*strat)() =
            get_random_strategy()
        strat();
    }
    return 0;
}

int main(void) {
    return save_the_pandas();
}
\end{lstlisting}
\end{minipage}

\begin{figure}
    \centering
    \begin{tikzpicture}[scale=1.0]
        \tikzset{vertex/.style = {shape=circle,draw,minimum size=8em}}
        \tikzset{edge/.style = {->,> = latex'}}
        % vertices
        \node[vertex] (main) at  (0, 0) {\lstinline{main}};
        \node[vertex, fill=green!50]
            (savethepandas) at  (5, 0) {\lstinline{save_the_pandas}};
        \node[vertex] (pandasaresaved) at (10, 0) {\lstinline{pandas_are_saved}};
        \node[vertex, shape=cloud, pattern=horizontal lines, pattern color=red!20]
            (strat) at (5, -4) {\lstinline{strat}};
        \node[vertex] (stopdeforestation) at  (5, -8) {\lstinline{stop_deforestation}};
        \node[vertex] (stophunting) at (10, -6) {\lstinline{stop_hunting}};
        \node[vertex, pattern=horizontal lines, pattern color=red!20]
            (breedandreleasepandas) at (0, -6)
            {\lstinline{breed_and_release_pandas}};
        \node[vertex] (releasepanda) at (0, -12) {\lstinline{release_panda}};
        \node[vertex, fill=red!50] (malloc) at (5, -12) {\lstinline{malloc}};
        %edges
        \draw[edge, -triangle 90, line width=0.5mm] (main) to (savethepandas);
        \draw[edge, -triangle 90, line width=0.5mm] (savethepandas) to (strat);
        \draw[edge, -triangle 90, line width=0.5mm] (savethepandas) to (pandasaresaved);
        \draw[edge, -triangle 90, line width=0.5mm] (breedandreleasepandas) to (releasepanda);
        \draw[edge, -triangle 90, line width=0.5mm] (breedandreleasepandas) to (malloc);
    \end{tikzpicture}
    \caption{Color-coded Call Graph for Listing \ref{lst:rules:pandaptrannote}.  Functions tagged \lstinline{static_memory} are highlighted green and functions tagged \lstinline{dynamic_memory} are highlighted red.  Indirect types are represented as horizontal line patterns on a node.  Clouds represent function pointers.}
    \label{fig:rules:pandaptrannote}
\end{figure}

Thanks to the graph based representation of this program, it is clear to see where the error is.  \lstinline{save_the_pandas} calls \lstinline{strat} and it is possible that a call to \lstinline{strat} could result in a call to \lstinline{malloc}.  The indirect type of \lstinline{strat} (notated in Figure \ref{fig:rules:pandaptrannote} as red horizontal lines) is how we keep track of this possibility.  

%\subsection{Indirect Type Inference}
%
%As mentioned earlier, funqual is able to infer the indirect type of functions.  For any function, $F$, the indirect type can be determined by traversing the call graph and visiting all the vertices that are reachable from $F$.  The indirect type of $F$ is the union of the direct types and indirect types of all the vertices that $F$ can reach.  
%
%Once again, the indirect type of $F$ is a list of types that could \textit{possibly} be reached from a call to $F$.  If we are trying to enforce a $restrict\_indirect\_call(X, Y)$ rule, it is important that under no circumstance, following no execution path, is it possible for a function in $X$ to call a function in $Y$.  Taking this expansive view of indirect type is the only way to guarantee call graph safety.  
%
%This indirect type inference is only possible because the body of the function can be found and traversed statically.  For function pointers, there is no way of knowing which functions the pointer can reference.  As a result, indirect type for function pointers must be annotated explicitly and every assignment into the function pointer is checked to ensure that the assignment does not lose any type information.

\subsection{Rules of Assignment}

To properly enforce call graph constraints, funqual checks function pointers in two places:  first when the function pointer is assigned, and second when the function pointer is called.  The rules described in this section are crafted specifically to maintain call graph correctness.  For the purpose of this discussion, we will let $L$ stand for some function pointer and we will let $R$ stand for some function value (the names $L$ and $R$ are a reference to the \lstinline{lvalue} and \lstinline{rvalue} in a typical assignment statement).  

When assigning a function pointer $L$ to point to a function $R$, there are two rules that funqual checks:  The direct type of $L$ must match exactly the direct type of $R$, and the indirect type of $L$ must be the superset of the indirect type of $R$.  For function pointers, both the direct and indirect types must be explicitly annotated in code.  For functions, only the direct type must be explicitly annotated as the indirect type can be inferred.  

These rules are necessary to maintain the soundness of the system.  In order to correctly enforce $require\_direct\_call(X, Y)$, the direct type of $L$ must be contained in $R$ --- otherwise a call to $L$ might be considered valid even if $R$ does not have $Y$ in its type.  In order to correctly enforce $restrict\_direct\_call(X, Y)$, the direct type of $R$ must be contained in $L$ --- otherwise a call to $L$ might be considered valid even if $R$ does not have $Y$ in its type.  Combining both of these requirements means that the direct types of $L$ and $R$ must match exactly.  Lastly, in order to properly enforce $restrict\_indirect\_call(X, Y)$, we need to know all the funqual types that are possibly reachable by calling $L$.  

Table \ref{fig:rules:assignment_table} shows a few examples of valid and invalid assignments.  

\begin{table}
    \centering
    \begin{tabular}{|c|c|c|c|c|}
        \hline
        lvalue & lvalue & rvalue & rvalue & Valid? \\
        direct & indirect & direct & indirect & \\
        \hline
        \hline
        \rowcolor{tablegreen}
        (none) & (none) & (none) & (none) & Valid \\
        \hline
        \rowcolor{tablered}
        static\_memory & (none) & (none) & (none) & Not Valid \\
        \hline
        \rowcolor{tablered}
        (none) & (none) & static\_memory & (none) & Not Valid \\
        \hline
        \rowcolor{tablegreen}
        static\_memory & (none) & static\_memory & (none) & Valid \\
        \hline
        \rowcolor{tablegreen}
        static\_memory & blocking & static\_memory & (none) & Valid \\
        \hline
        \rowcolor{tablered}
        static\_memory & (none) & static\_memory & blocking & Not Valid \\
        \hline
        \rowcolor{tablegreen}
        static\_memory & blocking & static\_memory & blocking & Valid \\
        \hline
        \rowcolor{tablered}
        static\_memory & blocking & static\_memory & nonblocking & Not Valid \\
        \hline
        \rowcolor{tablegreen}
        static\_memory & blocking & static\_memory & nonblocking & Valid \\
        \rowcolor{tablegreen}
         & nonblocking &  & & \\
        \hline
        \rowcolor{tablegreen}
        (none) & blocking & (none) & (none) & Valid \\
        \rowcolor{tablegreen}
         & static\_memory & & & \\ 
        \rowcolor{tablegreen}
         & nonblocking & & & \\ 
        \hline
    \end{tabular}
    \caption{Examples of valid and invalid assignments in funqual.  The left two collumns show the direct and indirect type of the lvalue respectively.  The next two columns show the direct and indirect type of the rvalue respectively.  The rightmost column shows whether or not that assignment is valid.}
    \label{fig:rules:assignment_table}
\end{table}




\section{Call Graph Rules}\label{sec:rules:rules}

% X and Y are type qualifiers (sets of functions)
% E is set of edges
% P is set of paths
% A and B are functions

Each subsection here describes one of the call graph rules supported by funqual.  For each rule, we explain the meaning, provide an algorithm that could enforce it, and present an argument for the algorithm's correctness with respect to the rest of the type system.  The algorithms presented here only return \lstinline{true} or \lstinline{false} depending on whether the graph in question is valid.  The algorithms actually implemented in funqual are slightly more complicated because they print helpful diagnostic messages to the user.  Both sets of algorithms enforce the same rules, though.  

\subsection{Restrict Direct Call}

\begin{center}
    $restrict\_direct\_call(X, Y)$
\end{center}

A restrict direct call rule creates a constraint that disallows functions with direct type $X$ from calling functions with direct type $Y$.  This constraint is relatively permissive because it still allows indirect calls from functions with direct type $X$ to functions with direct type $Y$ but is nonetheless checkable by this type system.

Figure \ref{lst:rules:rules:restrict_direct_call} shows pseudocode for an algorithm that can check a call graph for violations of this rule.  Assume that \lstinline{edges} is a list of objects representing all the calls in the call graph.  

\begin{figure}
    \begin{lstlisting}[gobble=8]
        function enforce_restrict_direct_call(X, Y, edges):
            for edge in edges:
                callee = edge.to
                caller = edge.from

                if X in caller.direct_type and Y in callee.direct_type:
                    return false
            return true
    \end{lstlisting}
    \caption{Pseudocode for an algorithm that can check a $restrict\_direct\_call$ constraint.  This algorithm returns \lstinline{true} if the call graph respects the constraint and \lstinline{false} if the call graph violates it.}
    \label{lst:rules:rules:restrict_direct_call}
\end{figure}

This algorithm runs once per rule and terminates in linear time with respect to the number of edges in the call graph.  To assert the correctness of this algorithm we will categorize each function call in this graph as one of two possibilities:  a call to a standard function, or a call to a function pointer.

\interfootnotelinepenalty=10000

In the case of a standard function call, the correctness is trivial.  The user must have annotated the direct type of both the caller and the callee\footnote{Funqual will check whatever was declared by the programmer --- whether the programmer declared their intent correctly is outside the scope of this research.}.  If a function with direct type $X$ calls a function with direct type $Y$, then \lstinline{edges} will contain such an edge and in checking each edge we will detect it.  

In the case of the function pointer call, we need to also examine all possible assignments of that function pointer.  It is of course possible that the function pointer is null at runtime, but we will consider this type of error to be out of the scope of funqual.  For the sake of this argument, let $P$ stand for any function pointer and $F$ stand for any function.  For an assignment of $F$ into $P$ to be valid, $F$ and $P$ must have the same direct type.  If they do not have the same direct type, then funqual will inform the user of an assignment type violation.  If they do have the same direct type, then \lstinline{edges} will contain an edge into $P$ wherever $P$ is called and that edge will be checked in the same way as a standard function call.  

\subsection{Restrict Indirect Call}

\begin{center}
    $restrict\_indirect\_call(X, Y)$
\end{center}

A restrict indirect call rule creates a constraint that functions with direct type $X$ cannot call functions with direct or indirect type $Y$.  This has the effect of restricting functions with direct type $X$ from calling functions with direct type $Y$, whether that call is direct or indirect.  The need to enforce indirect calls in the presence of function pointers requires us to examine the indirect type of the callee for each edge.  

\begin{figure}
    \begin{lstlisting}[gobble=8]
        function enforce_restrict_indirect_call(X, Y, edges):
            for edge in edges:
                callee = edge.to
                caller = edge.from

                if X in caller.direct_type and Y in callee.indirect_type:
                    return false
            return true
    \end{lstlisting}
    \caption{Pseudocode for an algorithm that can check a $restrict\_indirect\_call$ constraint.  This algorithm returns \lstinline{true} if the call graph respects the constraint and \lstinline{false} if the call graph violates it.}
    \label{lst:rules:rules:restrict_indirect_call}
\end{figure}

Figure \ref{lst:rules:rules:restrict_indirect_call} shows pseudocode for an algorithm that can check a call graph for violations of this rule.  Assume that \lstinline{edges} is a list of objects representing all the calls in the call graph.  

In order to simplify this algorithm, we will assume for the time being that indirect function types are inferred correctly.  For an explanation of the indirect type inference algorithm and for an argument for its correctness, refer to Subsection \ref{sec:rules:rules:inference}.  To assert the correctness of \lstinline{enforce_restrict_indirect_call}, we will again consider each function call in the graph as a member of one of two categories: a call to a standard function, or an invocation of a function pointer.

In the case of a standard function call, the correctness is trivial.  Assume function $A$ with direct type $X$ calls function $B$ with indirect type $Y$.  Since $A$ directly calls $B$, we know that there will be an edge from $A$ to $B$ in the \lstinline{edges} and when the algorithm visits it, the algorithm will terminate with the claim that there is a violation.

In the case of a function pointer invocation, the rules of function pointer assignment come into play.  If, via an invocation of $B$, a function of type $Y$ could eventually be called, then the function pointer must necessarily have $Y$ in its indirect type otherwise there would be an assignment error (for an in-depth argument of this refer to Subsection \ref{sec:rules:rules:inference}).  As a result, when visiting the edge from $A$ to $B$ (where $A$ is the function invoking function pointer $B$), the algorithm will detect that $B$ has indirect type $Y$ and will terminate with the claim that there is a violation.  

\subsection{Require Direct Call}

\begin{center}
    $require\_direct\_call(X, Y)$
\end{center}

A require direct call rule creates a constraint that functions with direct type $X$ can only call functions with direct type $Y$. Much like the restrict direct call rule, this rule is relatively easy to check and can be checked in time linear with respect to the number of edges in the call graph.

Figure \ref{lst:rules:rules:require_direct_call} shows pseudocode for an algorithm that can check a call graph for violations of this rule.  Assume that \lstinline{edges} is a list of objects representing all the calls in the call graph.  

\begin{figure}
    \begin{lstlisting}[gobble=8]
        function enforce_require_direct_call(X, Y, edges):
            for edge in edges:
                callee = edge.to
                caller = edge.from

                if X in caller.direct_type and Y not in callee.direct_type:
                    return false
            return true
    \end{lstlisting}
    \caption{Pseudocode for an algorithm that can check a $require\_direct\_call$ constraint.  This algorithm returns \lstinline{true} if the call graph respects the constraint and \lstinline{false} if the call graph violates it.}
    \label{lst:rules:rules:require_direct_call}
\end{figure}

To assert the correctness of this algorithm, we will categorize every function call as one of two possibilities: a call to a standard function, or a call to a function pointer.

In the case of a call to a standard function, the correctness is trivial.  The user must have annotated the direct type of both the caller and the callee and we take these annotations to be correct.  If a function with direct type $X$ calls any function, then \lstinline{edges} will contain an edge from the caller to the callee.  Checking the direct types of caller and callee exhaustively for every edge in the graph will eventually find any violations.

In the case of a function pointer call, we need to also examine all the possible assignments to that function pointer.  Thankfully the assignment checker already checked the type safety of every function pointer assignment so we will assume that those are correct.  In this case specifically, we can assume that, if the function which is actually called does not have direct type $Y$, then the function pointer which is called in code will also not have direct type $Y$.  This call creates an edge which will certainly be visited by \lstinline{enforce_restrict_direct_call} and so we can be certain that any function pointer invocation will be correctly checked in this regard.

\subsection{Indirect Type Inference}\label{sec:rules:rules:inference}

While the user does not invoke indirect type inference in the same way that the user invokes the other rules, indirect type inference is still an important part of the type safety of funqual.  This subsection explains indirect type inference and argues for the correctness of the algorithm.  

Figure \ref{lst:rules:rules:infer_indirect_type} shows pseudocode for an algorithm that can infer the indirect function type for any function in the call graph.  For the purpose of this function, we will let \lstinline{function} be the function being checked.  We will let \lstinline{edges} be the list of edges in our graph and we will assume that it contains edges to function pointers where those function pointers are called.  We also assume that \lstinline{callee.indirect_type} is populated for function pointers but that it is an empty set for regular functions.  

\begin{figure}
    \begin{lstlisting}[gobble=8]
        function infer_indirect_type(function, edges):
            indirect_types = empty set
            visited = empty set
            to_visit = empty set
            to_visit.add(function)

            while to_visit is not empty:
                curr = to_visit.pop()
                visited.add(curr)

                indirect_types.add_all(curr.direct_type)
                indirect_types.add_all(curr.indirect_type)

                for edge in edges:
                    callee = edge.to
                    caller = edge.from
                    if caller == curr and callee not in visited:
                        to_visit.add(callee)
            return indirect_types
    \end{lstlisting}
    \caption{Pseudocode for an algorithm to infer the indirect type of a function.}
    \label{lst:rules:rules:infer_indirect_type}
\end{figure}

To assert the correctness of this algorithm imagine a function, $F$, from which evaluation eventually (either directly or indirectly) reaches a function, $C$, with type $Y$.  We propose that because of the rules of this type system, it is necessary that $Y$ is in the type of $F$.  To demonstrate this we will break down the type pipeline into its multiple cases.

The first case is that $F$ calls $C$ (either directly or indirectly) but that none of the calls from $F$ to $C$ are function pointer invocations.  In this case, there will be a path in \lstinline{edges} from $F$ to $C$ and because \lstinline{infer_indirect_type} is a breadth first graph traversal starting at $F$, we know that the algorithm will eventually visit $C$.  When the algorithm does visit $C$, it will grab the direct type of $C$ (which contains $Y$) and add it to the indirect type of $F$.  When the algorithm terminates, it will necessarily contain $Y$.  In other words, if there is a path from $F$ to $C$, the indirect type of $F$ will contain the direct and indirect types of $C$.

The second case is that $F$ invokes a function pointer $P$ from which evaluation eventually results in a call to $C$.  In this case, there may or may not be a path in \lstinline{edges} from $F$ to $C$.  However, there will be a path in \lstinline{edges} from $F$ to $P$ and an assignment of $C$ into $P$.  Recall that for an assignment of $C$ into $P$ to typecheck, the direct types of $C$ and $P$ must match and the indirect type of $P$ must contain the indirect type of $C$.  If $Y$ is in the direct type of $C$, then $Y$ must be in the direct type of $P$.  Also if $Y$ is in the indirect type of $C$, then $Y$ must be in the indirect type of $P$.  Since either the direct type or the indirect type of $P$ must contain $Y$, we can reference case one and claim that because there is a path from $F$ to $P$, and because the type of $P$ contains $Y$, then $Y$ will be in the indirect type of $F$.  

The third case is an inductive step.  Assume that $F$ calls $C$ but indirectly through some arbitrary number of function pointer invocations between.  Let $P_0$ be a function pointer through which a call is made to $C$, let $P_1$ be a function pointer through which a call is made to $P_0$, let $P_n$ be a function pointer through which a call is made to $P_{n-1}$, and let $F$ call $P_n$.  According to the logic in case two, if $Y$ is in the direct type of $C$, then it must be in the direct type of $P_0$ or else the assignment will have failed.  In the same way, if $Y$ is in the direct or indirect type of $P_{n-1}$, then it must be in the direct or indirect type of $P_{n}$.  Inductively, $Y$ must be in the direct or indirect type of $P_n$ and because there is a path in \lstinline{edges} from $F$ to $P_n$, $Y$ must end up in the indirect type of $F$.  Lastly, as in case one, any of these calls (either from $F$ to $P_n$, from $P_n$ to $P_{n-1}$, or from $P_0$ to $C$) can be direct or indirect calls and $Y$ will still be in the indirect type of $F$.

This algorithm terminates even in the presence of cycles because it tracks previously visited vertices in \lstinline{visited} and does not visit them again.  Even though these cyclic edges are not followed, the output is still correct because every vertex is visited once.  Assume that $F$ calls $C$ and that $C$ calls $F$.  The algorithm first visits $F$, then visits $C$, but does not visit $F$ again because $F$ was added to \lstinline{visited} when it was first examined.  When $F$ was added to \lstinline{visited}, its direct and indirect type were added to the return value and the edges out of $F$ were added to \lstinline{to_visit}.  All the necessary information was extracted from $F$ on the first visit so visiting it again is not necessary.


\section{Special Considerations when Creating a Call Graph}\label{sec:rules:special}

\subsection{Dealing with Inheritance}\label{sec:rules:inherit}

When calling a virtual method in C++, it is impossible to know at compile time exactly which function is going to be run at run-time.  This is very similar to the problem of function pointers (and in fact dynamic dispatch is usually implemented as a table of function pointers \cite{language-standard}) except that in the case of virtual functions we actually know statically the set of possible functions that could be called\footnote{Funqual assumes that it has access to the full source code for call graph creation.}.  To account for this, we need to add extra edges to our call graph to represent all the possible places that a virtual method call could go.  

Let $C$ be some function that calls $T.M$ where $T$ is some class and $M$ is a virtual method of $T$.  When creating the call graph, we must surely add an edge from $C$ to $T.M$.  In addition to that, though, for any class $S$ that is a subclass of $T$, we must also add an edge from $C$ to $S.M$.  This accounts for any possible overloads of $M$ that might be called at run-time.

Figure \ref{lst:rules:inheritance} demonstrates this concept.  It is a piece of C++ source code that calls a virtual function.  Figure \ref{fig:rules:inheritance} shows the call graph for this code sample.  

\begin{figure}
    \begin{lstlisting}[language=C++,gobble=8]
        class Panda {
        protected:
            int m_hunger;
        public:
            virtual int Feed() {
                m_hunger--;
            }
        };

        class RedPanda : public Panda{
        public:
            int Feed() override {
                Stomach *stomach = malloc(sizeof(Stomach));
                memset(stomach, 0xFF, sizeof(Stomach));
            }
        };

        void feedPanda(Panda *panda) static_memory {
            panda->Feed();
        }

        int main(void) {
            feedPanda(new RedPanda());
        }
    \end{lstlisting}
    \caption{Example C++ program demonstrating inheritance.  In \lstinline{feedPanda}, it is impossible to know statically which instance of the \lstinline{Feed} function will be called.  Figure \ref{fig:rules:inheritance} shows the call graph for this program.}
    \label{lst:rules:inheritance}
\end{figure}

\begin{figure}
    \centering
    \begin{tikzpicture}[scale=1.0]
        \tikzset{vertex/.style = {shape=circle,draw,minimum size=8em}}
        \tikzset{edge/.style = {->,> = latex'}}
        % vertices
        \node[vertex] (main) at  (0, 0) {\lstinline{main}};
        \node[vertex, fill=green!50] (feedPanda) at  (5, 0) {\lstinline{feedPanda}};
        \node[vertex] (PandaFeed) at (0, -4) {\lstinline{Panda::Feed}};
        \node[vertex] (RedPandaFeed) at  (5, -4) {\lstinline{RedPanda::Feed}};
        \node[vertex] (memset) at  (10, -2) {\lstinline{memset}};
        \node[vertex, fill=red!50] (malloc) at  (10, -6) {\lstinline{malloc}};
        %edges
        \draw[edge, -triangle 90, line width=0.5mm] (main) to (feedPanda);
        \draw[edge, -triangle 90, line width=0.5mm] (feedPanda) to (PandaFeed);
        \draw[edge, -triangle 90, line width=0.5mm] (feedPanda) to (RedPandaFeed);
        \draw[edge, -triangle 90, line width=0.5mm] (RedPandaFeed) to (malloc);
        \draw[edge, -triangle 90, line width=0.5mm] (RedPandaFeed) to (memset);
    \end{tikzpicture}
    \caption{Call graph for Figure \ref{lst:rules:inheritance}.  Because \lstinline{Panda::Feed} is a virtual function, we must draw an edge from \lstinline{feedPanda} to every instance of \lstinline{Feed}.}
    \label{fig:rules:inheritance}
\end{figure}

For this example we will continue to assume that there is a rule restricting indirect calls from \lstinline{static_memory} functions to \lstinline{dynamic_memory} functions.  In \lstinline{feedPanda} we can see that we call \lstinline{Panda::Feed}.  This is somewhat misleading:  \lstinline{Panda::Feed} is a virtual function and it is overridden by a child class called \lstinline{RedPanda}.  This means that any time \lstinline{feedPanda} is called, it is impossible to know whether it is \lstinline{Panda::Feed} being called or whether it is actually \lstinline{RedPanda::Feed} being called.  The only safe way to handle this scenario is to assume that \lstinline{feedPanda} calls both of them.  This is reflected in Figure \ref{fig:rules:inheritance} which is a call graph showing \lstinline{feedPanda} pointing to both versions of the \lstinline{Feed} function.  

\subsection{Overriding Methods with Annotations}

Just like with the standard C++17 type qualifiers, if a virtual function $T.M$ is overridden by $S.M$, then the qualified types of $T.M$ and $S.M$ must match exactly.  Figure~\ref{lst:rules:bad_override} contains an example of an override that is invalid according to the C++17 standard.  In this example, \lstinline{Panda::Feed} has a \lstinline{const} qualifier, but \lstinline{RedPanda::Feed} does not.  As such, the two functions have different types and the compiler will generate an error.  

\begin{figure}
    \begin{lstlisting}[language=C++,gobble=8]
        class Panda {
        public:
            virtual void Feed() const;
        };

        class RedPanda: public Panda {
            virtual void Feed() override;
        };
    \end{lstlisting}
    \caption{Example C++ containing an error.  \lstinline{Panda::Feed} and \lstinline{RedPanda::Feed} have different types and so the override is invalid.}
    \label{lst:rules:bad_override}
\end{figure}

Funqual treats funqual direct type in the same way.  For $T.M$ to be overridden by $S.M$, the two functions must have the same direct type.  If they do not, funqual will display an error.  The Figure~\ref{lst:rules:bad_override_qtag} shows a similar example of an invalid override but where the direct type is the type in conflict.

\begin{figure}
    \begin{lstlisting}[language=C++,gobble=8]
        class Panda {
        public:
            virtual void Feed() QTAG(static_memory);
        };

        class RedPanda: public Panda {
            virtual void Feed() override;
        };
    \end{lstlisting}
    \caption{Example C++ containing a funqual type error.  \lstinline{Panda::Feed} and \lstinline{RedPanda::Feed} have different types and so the override is invalid.}
    \label{lst:rules:bad_override_qtag}
\end{figure}

\subsection{Operator Overloading}

C++ allows for operator overloading.  As a result, an expression such as \mbox{\lstinline{a = b + c;}} could result in a function call depending on the types of \lstinline{a} and \lstinline{b}.  

Compensating for this is relatively straightforward.  When funqual comes across a binary or unary operator that can be overloaded, it checks the type of the operand(s) and checks for an operator overload.  If there is an operator overload, then the call graph will contain an edge from the calling context to the overload function.  If the overload is virtual, funqual checks for operator overloads in child classes as described in Section \ref{sec:rules:inherit}.

\subsection{Bridging the Divide between Translation Units}

The compilation of C++ code is driven by translation units.  Translation units are the files which are provided to the C compiler to be translated into object files.  In general, translation units are singular \lstinline{.c} or \lstinline{.cpp} files including any source files that may be \lstinline{#include}-ed.  During this process, many symbols are said to have \textit{external linkage} meaning that their type is specified in this translation unit but that their value is not (this is the case with extern variables, function prototypes, and class forward declarations).  In these cases, examining the call graph of a single translation unit is not sufficient to enforcing global call graph constraints because we would not be able to see the calls made in other translation units which may be of interest for enforcing indirect call restrictions.  

To solve this problem we need to examine every translation unit in the source and build a call graph that represents the entire codebase.  In order to test this, we create several test cases where functions are defined in multiple translation units and where a function call graph constraint is violated between translation units.

