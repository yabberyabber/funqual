\chapter{Application}\label{sec:application}

This chapter demonstrates three real use-cases for funqual.  In each section below, we outline the application of funqual to a project, the constraint that funqual was used to enforce, and the outcome of using funqual to check that constraint.  \mbox{Section \ref{sec:app:reentrancy}} describes using funqual to prevent reentrancy errors in a class assignment for Operating Systems at Cal Poly (CSC453).  Section \ref{sec:app:pre-malloc} describes using funqual to prevent use of \lstinline{malloc} and \lstinline{printf} during boot-up of a custom kernel written in a class assignment for Operating Systems 2 at Cal Poly (CSC454).  Lastly, \mbox{Section \ref{sec:app:blocking}} describes using funqual to prevent the use of potentially blocking calls in high frequency loops in a robotics application.  

All of these projects were developed before funqual existed, so funqual was not used during the development cycle.  The goal of this chapter is to demonstrate that funqual can scale beyond small test cases and to demonstrate how funqual can be used to address a variety of real-world issues.  

\section{Glibc Nonreentrant Functions}\label{sec:app:reentrancy}

\interfootnotelinepenalty=10000

The GNU C Library Reference Manual warns against calling nonreentrant functions from signal handlers \cite{gnu-manual}.  A function which only accesses memory within its stack frame is reentrant because it cannot be affected by external state.  A function which accesses heap, global, or static memory may be nonreentrant if that memory can be modified by other execution environments.  This includes functions which reference a global datastructure (e.g. \lstinline{malloc}) or grab a global lock (e.g. \lstinline{printf}).  Reentrancy is a separate but similar concept to thread-safety; a reentrant function is thread-safe but a thread-safe function may not necessarily be reentrant.  As an example, \lstinline{printf} could be considered thread safe because it locks the stream while writing to it.  However, if a call to \lstinline{printf} is interrupted while it holds the lock and the interrupt handler makes its own call to \lstinline{printf}, then the interrupt handler will wait for the lock.  Since the code holding the lock cannot run until the interrupt handler finishes, the system is in deadlock.  This is bad --- we would like to prevent this error as well as errors like it.  

Funqual can find and report this type of error.  To demonstrate this, we take a class assignment written for an Operating Systems class (CSC453) that uses signal handlers, insert function type qualifiers, and create a rules file.  The assignment was to simulate a set of snakes crawling around the screen.  Each time the user presses control-C (creating a \lstinline{SIGINT} signal), one of the snakes disappears.  When the user tries to kill the process (creating a \lstinline{SIGQUIT} signal), the program makes each snake disappear and then terminates.  If a signal is sent during a call to a nonreentrant function, that function is preempted by the signal handler; if the signal handler calls that same nonreentrant function, this can result in undefined behavior.  

To make funqual detect this issue, we use two type qualifiers: \lstinline{preemptive} which applies to signal handlers, and \lstinline{non_reentrant} which applies to nonreentrant functions.  We also create one rule: $restrict\_indirect\_call(preemptive, non\_reentrant)$.  Since many of the nonreentrant functions we are concerned about are in the C standard library, these functions are annotated as \lstinline{non_reentrant} in the rules file.  Figure~\ref{lst:app:reentrancy:rules} shows the rules file used.  The list of functions tagged as nonreentrant is incomplete but represents the ones used in this program.  In addition to tagging nonreentrant library functions in the rules file, the signal handlers in the code are tagged as \lstinline{preemptive}.  Figure~\ref{lst:app:reentrancy:handlers} shows the two lines that were added to the program source to tag signal handlers.  

\begin{figure}
    \begin{lstlisting}[gobble=8]
        rule restrict_indirect_call preemptive non_reentrant

        tag c:@F@malloc non_reentrant
        tag c:@F@free non_reentrant
        tag c:@F@printf non_reentrant
        tag c:@F@fprintf non_reentrant
        tag c:@F@sprintf non_reentrant
        tag c:@F@rand non_reentrant
    \end{lstlisting}
    \caption{Rules file for preventing preemptive functions from calling non\_reentrant functions.  Since this rules file contains no references to project-specific functions, the file could conceivably be re-used by several projects.}
    \label{lst:app:reentrancy:rules}
\end{figure}

\begin{figure}
    \begin{lstlisting}[language=c,gobble=8]
        void kill_snake() QTAG(preemptive);
        void lwp_stop() QTAG(preemptive);
    \end{lstlisting}
    \caption{Lines inserted into C file to mark signal handlers as preemptive.}
    \label{lst:app:reentrancy:handlers}
\end{figure}

The size of this project was 458 lines of code\footnote{Line count achieved using the \lstinline{cloc} utility not including comments or blank lines.} contained in five \lstinline{.c} files with 40 edges in the call graph.  Funqual analyzed the source in about 0.9 seconds\footnote{Data collected on a T460 Lenovo Thinkpad with Quad Intel Core i5-6300U CPU at 2.4GHz.} --- 0.1 seconds were spent in libClang parsing the source, 0.8 seconds were spent traversing the AST to generate the call graph, 0.001 seconds were spent performing type inference, 0.000,01 seconds were spent checking the call graph, and 0.000,01 seconds were spent checking assignments. 

On the first run, funqual did not detect any call graph violations.  In order to test that the tool does actually detect errors, several illicit calls to \lstinline{printf} were inserted.  After doing so, funqual correctly detected and reported these errors.  Figure~\ref{lst:app:reentrancy:output} shows the output from funqual when run on this modified codebase.

\begin{figure}
    \begin{lstlisting}[gobble=8]
        Rule violation: `non_reentrant` function indirectly called from `preemptive` context
                Path:   snakemain.c::lwp_stop(int) (68,14)
                -calls: libLWP.c::get_init_context() (193,6)
                -calls: libLWP.c::printf(const char *, ...) (362,12)
    \end{lstlisting}
    \caption{Output from funqual when run on a project that had manually-inserted call graph violations}
    \label{lst:app:reentrancy:output}
\end{minipage}

As seen in Figure~\ref{lst:app:reentrancy:output}, funqual successfully found a call graph violation that was manually inserted into the program source.  Additionally, funqual listed the locations in code where each call occurred between the \lstinline{preemptive} and \lstinline{non_reentrant} functions.  




\section{Restricting API available during kernel initialization}\label{sec:app:pre-malloc}

Kernel development is complicated for a variety of reasons.  One reason that makes it particularly complicated is that not all of the standard libraries are available from certain contexts in the kernel.  Just like in Section \ref{sec:app:reentrancy}, it would be a serious issue if we were to call \lstinline{malloc} or \lstinline{printf} from within an interrupt handler.  In addition to that, we also need to ensure that these functions are not called before their associated interfaces are initialized.  As an example, the use of \lstinline{malloc} depends on the page table having been initialized, and so there are contexts within kernel initialization where a call to \lstinline{malloc} would be inappropriate.  We would like to prevent this error as well as errors like it. 

To demonstrate how funqual can be applied to this problem, we take an assignment for Operating Systems II (CSC454) where students develop their own simple x86\_64 kernel, and augment it with function type qualifiers and a rules file.  The kernel consists of several subsystems that are each initialized in sequence.  These subsystems, in order, are: a VGA subsystem to display text on the screen, a PC2 subsystem to poll the keyboard and mouse for input, a subsystem to schedule interrupts and register interrupt handlers, an interface to send and receive text over the system's serial interface, a memory manager to allocate physical pages and add them to the page table, a scheduler to run multiple processes, and lastly a set of processes which each draw a snake crawling around the screen like in Section \ref{sec:app:reentrancy}.  These subsystems all depend on things in the previous subsystems.  The keyboard subsystem depends on the VGA subsystem to display the keys the user pressed, the physical memory manager depends on the interrupt subsystem to listen for page faults, \lstinline{malloc} depends on the memory manager, and so on.  It is very easy in the early stages of boot-up to accidentally call a function in a subsystem that has not been initialized.  Sometimes these calls are hidden by a few edges in the call graph, making it hard for a human to detect them.  

To solve this problem, we use several type qualifiers:  \lstinline{pre_vga}, \lstinline{vga}, \lstinline{pre_pc2}, \lstinline{pc2}, \lstinline{pre_irq}, \lstinline{irq}, \lstinline{pre_ser}, \lstinline{ser}, \lstinline{pre_mmu}, \lstinline{mmu}, \lstinline{pre_proc}, and \lstinline{proc}.  We also create rules for each subsystem that prevent functions tagged \lstinline{pre_XXX} from calling functions in any of the subsystems that involve \lstinline{XXX}.  Listing \ref{lst:app:pre-malloc:rules} shows the rules file used to support this.  

The subsystems are initialized in the following order: VGA, PC2, Interrupt Request (IRQ), Serial, Memory Management Unit (MMU), and process manager (proc).  For each subsystem, there is an \lstinline{_init} function which is called to set the subsystem up.  The \lstinline{_init} function for each subsystem is annotated with \lstinline{pre_XXX} for each subsystem that it precedes.  Listing \ref{lst:app:pre-malloc:annote} shows the annotations that were added to accomplish this.

\noindent\begin{minipage}[t]{\linewidth}
\begin{lstlisting}[caption={Rules file for a simple kernel written for CSC454.  The rules written here are intended to prevent code which runs before a subsystem is initialized from calling any function that depend on the subsystem.},label={lst:app:pre-malloc:rules}]
rule restrict_indirect_call pre_vga vga
rule restrict_indirect_call pre_vga pc2
rule restrict_indirect_call pre_vga irq
rule restrict_indirect_call pre_vga ser
rule restrict_indirect_call pre_vga mmu
rule restrict_indirect_call pre_vga proc

rule restrict_indirect_call pre_pc2 pc2
rule restrict_indirect_call pre_pc2 irq
rule restrict_indirect_call pre_pc2 ser
rule restrict_indirect_call pre_pc2 mmu
rule restrict_indirect_call pre_pc2 proc

rule restrict_indirect_call pre_irq irq
rule restrict_indirect_call pre_irq ser
rule restrict_indirect_call pre_irq mmu
rule restrict_indirect_call pre_irq proc

rule restrict_indirect_call pre_ser ser
rule restrict_indirect_call pre_ser mmu
rule restrict_indirect_call pre_ser proc

rule restrict_indirect_call pre_mmu mmu
rule restrict_indirect_call pre_mmu proc

rule restrict_indirect_call pre_proc proc
\end{lstlisting}
\end{minipage}

% separator between 2 listings

\noindent\begin{minipage}[t]{\linewidth}
\begin{lstlisting}[language=c++,caption={Lines inserted into C source for a simple kernel in order to prevent subsystems from depending on interfaces not yet initialized.},label={lst:app:pre-malloc:annote}]
// for VGA subsystem
bool VGA_init() QTAG(pre_pc2);
void set_char_at(int row, int col, char character, char attributes)
    QTAG(vga) QTAG(pre_pc2);

// for PC2 subsystem
void PC2_init() QTAG(pre_irq);
bool get_char(char *ret) QTAG(pc2) QTAG(pre_irq);

// for interrupt request subsystem
void IRQ_init() QTAG(pre_ser);
void IRQ_set_handler(int irq, irq_handler_t handler, void *args)
    QTAG(irq) QTAG(pre_ser);

// for serial subsystem
void SER_init() QTAG(pre_mmu);
int SER_write(const char *buff, int len) QTAG(ser) QTAG(pre_mmu);

// for memory management unit subsystem
void MMU_pt_init() QTAG(pre_proc);
void *MMU_alloc_page() QTAG(mmu) QTAG(pre_proc);
void MMU_free_page(void *page) QTAG(mmu) QTAG(pre_proc);

// for multiprocessing subsystem
void PROC_init();
Process_t *PROC_create_kthread(kproc_t entry_point, void *arg)
    QTAG(proc);
void PROC_run() QTAG(proc);
\end{lstlisting}
\end{minipage}

This project contained 6034 lines of C code spread over 36 files with 291 edges in the call graph\footnote{Line count achieved using the \lstinline{cloc} utility not including comments or blank lines.  Many of these lines were machine-generated.}.  Funqual analyzed this source in about 2.2 seconds --- 0.3 seconds were spend in libClang parsing source files, 1.9 seconds were spent building the call graph, 0.005 seconds were spent performing type inference, less than 0.000,05 seconds were spent checking the call graph, and less than 0.000,05 seconds were spent checking function pointer assignments.  

Funqual did detect errors in this source.  Listing \ref{lst:app:pre-malloc:output} shows the output from running funqual.  According to this output, funqual detected several cases where a \lstinline{pre_ser} function called a \lstinline{ser} function.  Specifically, \lstinline{IRQ_init} called \lstinline{printk} which eventually calls \lstinline{SER_write}.  \lstinline{printk} in this project operates just like \lstinline{printf}.  The only difference is that \lstinline{printk} outputs to both VGA and to serial.  This rule violation represents an actual error that existed in the project which was not detected until funqual found it.  Ideally, \lstinline{SER_write} just fills a buffer; this buffer is emptied when the serial port sends an interrupt requesting data.  However, if the serial interface isn't operating, that buffer might fill up causing the caller to block until space becomes available in the buffer.  If we had filled the entire 1024 character buffer before initializing the serial interface, the kernel would have hanged indefinitely.

\noindent\begin{minipage}[t]{\linewidth}
\begin{lstlisting}[caption={Output from funqual when run on simple OS kernel},label={lst:app:pre-malloc:output}]
Rule violation: `ser` function indirectly called from `pre_ser` context
        Path:   src/IRQ.c::IRQ_init() (12,6)
        -calls: src/io.c::printk(const char *, ...) (7,5)
        -calls: src/io.c::print_char(char) (9,6)
        -calls: src/main.c::SER_write(const char *, int) (7,5)

Rule violation: `ser` function indirectly called from `pre_ser` context
        Path:   src/IRQ.c::IRQ_init() (12,6)
        -calls: src/io.c::printk(const char *, ...) (7,5)
        -calls: src/io.c::print_str(const char *) (10,6)
        -calls: src/main.c::SER_write(const char *, int) (7,5)

Rule violation: `ser` function indirectly called from `pre_ser` context
        Path:   src/IRQ.c::IRQ_init() (12,6)
        -calls: src/io.c::printk(const char *, ...) (7,5)
        -calls: src/io.c::print_long_hex(uint64_t, bool) (13,6)
        -calls: src/io.c::print_str(const char *) (10,6)
        -calls: src/main.c::SER_write(const char *, int) (7,5)

Rule violation: `ser` function indirectly called from `pre_ser` context
        Path:   src/IRQ.c::IRQ_init() (12,6)
        -calls: src/io.c::printk(const char *, ...) (7,5)
        -calls: src/io.c::print_decimal(int64_t) (14,6)
        -calls: src/io.c::print_str(const char *) (10,6)
        -calls: src/main.c::SER_write(const char *, int) (7,5)
\end{lstlisting}
\end{minipage}

Observing the version history of this codebase tells the full story of how this bug occurred.  Initially, \lstinline{printk} only printed to VGA.  At that point in time, it was okay to call \lstinline{printk} from \lstinline{IRQ_init} because \lstinline{printk} only depended on VGA.  However, at some point a decision was made that \lstinline{printk} should print to both VGA and serial.  At this point, a simple modification was made to \lstinline{print_str} without considering all the ways that \lstinline{printk} was used. 

This sort of mistake is probably very familiar to many programmers and it might exist for quite a while before presenting as an error.  Imagine what the symptoms would have been: the programmer would have added an extra \lstinline{printk} to \lstinline{IRQ_init} and the programmer would have observed that the program hanged.  Why did it hang, though?  In kernel development, there are a lot of things that could cause a hang, but a simple call to \lstinline{printk} might not be the first thing one would check.  It would take quite a lot of digging to go from \lstinline{printk} to \lstinline{print_char} to \lstinline{SER_write} to then see that the issue was there.  

This situation really demonstrates the strength of funqual.  The programmer cannot possibly be cognisant of every location from which a function is called.  What's more, the programmer is usually unaware of paths between functions, especially when there are several levels of indirection in-between them.  Letting the programmer express their intuition as hard-coded rules and using a tool to check those rules automatically enables the programmer to be confident that errors like this don't occur in practice.  


\section{Detecting slow function calls in high frequency contexts}\label{sec:app:blocking}

In robotic motion control, proper timing is paramount to good performance.  In industrial automation, control loops\footnote{A control loop is an algorithm that measures some aspect of a system, calculates an output vector, and applies that output vector in a continuous loop until the system reaches a desired state.  For example, a car's cruise-control will measure the current speed, calculate how much gas to apply, and apply that much gas in a tight loop.  For each iteration, the process of taking input, calculating output, and applying output is called one \textit{cycle}.} often run on special hardware with real-time guarantees at several thousand cycles per second.  Specialized hardware like this is out of reach for robotics hobbyists, so we use general purpose hardware and GNU/Linux to achieve similar results. 

Without the real-time guarantees of a specialized environment, it is very difficult to maintain consistent control loops even at 100Hz.  Part of the problem is that the GNU/Linux ``real-time'' scheduler is subject to some minor jitter~\cite{rt-jitter}, but a lot of the problem stems from the fact that much of the standard libraries we generally use are not well-suited for soft real-time applications like this.  Over years of troubleshooting slow control loops, the software team at the Atascadero Education Foundation has created a list of functions which are sometimes slow and should never be called in these high frequency contexts.  We would like a tool to statically check that these high frequency contexts never call into functions in this list of slow functions.  

We use funqual to find and report these issues.  To do this, we use two type qualifiers: \lstinline{hi_freq} to represent these high-frequency control loops, and \lstinline{slow} to represent those functions which should not be called from within a control loop.  We also create one rule: $restrict\_indirect\_call(hi\_freq, slow)$ which tells funqual that \lstinline{hi_freq} functions should never call \lstinline{slow} functions whether directly or indirectly. 

An alternative approach which we abandoned because of increased burden on the programmer is to use just one function qualifier, \lstinline{fast}, and one rule, $require\_direct\_call(\allowbreak fast, fast)$, to require that each function marked \lstinline{fast} only call other functions marked \lstinline{fast}.  The issue with this approach is that  functions would be guilty until proven innocent --- there are hundreds of library functions that we do use in control loops without issue and each of these would need to be marked as \lstinline{fast} before funqual would accept them.  This increases the adoption cost for this approach and requires a lot of annotation. If we were more serious about guaranteeing close to real-time performance, this approach would have been more appealing.  

Listing \ref{lst:app:blocking:rules} shows the rules file that was written to accomplish this.  Several functions from various libraries are tagged in the rules file as \lstinline{slow} and there is one rule restricting \lstinline{hi_freq} functions from calling \lstinline{slow} functions.  This is not an exhaustive list of slow functions, but it is a representative subset of them.  

\noindent\begin{minipage}[t]{\linewidth}
\begin{lstlisting}[caption={Rules file for preventing high frequency functions from calling slow functions.  Several functions from standard libraries are marked in the rules file as \lstinline{slow}.},label={lst:app:blocking:rules}]
rule restrict_indirect_call hi_freq slow

tag c:@F@malloc slow
tag c:@F@printf slow
tag c:@F@fprintf slow
tag c:@F@sprintf slow
tag c:@N@frc@S@CameraServer@F@GetInstance#S slow
tag c:@N@frc@S@DriverStation@F@GetInstance#S slow
tag c:@N@frc@S@Scheduler@F@GetInstance#S slow
tag c:@N@nt@S@NetworkTableEntry@F@GetInstance#1 slow
tag c:@N@nt@S@NetworkTable@F@GetInstance#1 slow
\end{lstlisting}
\end{minipage}

In addition to creating the rules file which contains the rule and which marks several functions as \lstinline{slow}, we also modified several lines in the source to mark certain functions as \lstinline{hi_freq}.  Listing \ref{lst:app:blocking:code} shows the lines that were added to the C++ source file in order to achieve this.  This is not an exhaustive list of high frequency functions, but it is a representative subset of them.  

\noindent\begin{minipage}[t]{\linewidth}
\begin{lstlisting}[language=c++,caption={Lines inserted into C++ source file to mark certain functions as \lstinline{hi_freq}.},label={lst:app:blocking:code}]
void CoopMTRobot::DisabledPeriodic() QTAG(hi_freq) override;
void CoopMTRobot::AutonomousPeriodic() QTAG(hi_freq) override;
void CoopMTRobot::TeleopPeriodic() QTAG(hi_freq) override;
void CoopMTRobot::TestPeriodic() QTAG(hi_freq) override;

static void *SPIGyro::Run() QTAG(hi_freq);

bool SmartPixy::getStart() QTAG(hi_freq);
\end{lstlisting}
\end{minipage}

This codebase is big, in large part because of all the libraries we depend on for interfacing with sensors and actuators on the robot.  In the interest of time we could not check the entire codebase but rather we focused on the core libraries and the subsystems containing control loops.  The portion of the library that we checked consists of 6959 lines of C++ code spread out over 42 files.  The header files we include from other libraries consists of 12,506 lines of code spread out over 145 files.  As such, analyzing these files takes a long time.  Funqual analyzed the source in about 4 minutes --- 24 seconds were spent in libClang parsing the source, 209 seconds were spent traversing the AST building a call graph, 0.04 seconds were spent performing type inference, 0.000,02 seconds were spend checking the call graph, and less than 0.000,005 seconds were spent checking function pointer assignments.  The call graph contains 11635 vertices and 5103 edges.  Obviously due to the size of this project, it takes a long time for funqual to traverse it all.  Usually, projects using clang alleviate this by doing incremental compilation --- only the files which changed need to be examined.  Funqual does not currently support incremental linting (implementing it is certainly possible but would take significant development time).  If funqual did support incremental linting, then the time to run funqual on the codebase would be significantly reduced for most runs.  

Funqual found many errors in this codebase.  Because the funqual output is so large, the entire listing is not included here, but a representative portion of the output is shown in Listing \ref{lst:app:blocking:output}.  Most of these errors relate to stray debug \lstinline{printf}s inserted into the code.  

These calls were inserted for temporary debugging purposes and should definitely be removed.  Calls to \lstinline{printf} sometimes block for up to a few milliseconds when the output buffer gets filled and data needs to be copied somewhere else.  When a loop runs at 100Hz (10ms per cycle), a delay of a few milliseconds can slow the loop and degrade performance.  As such, these rule violations represent actual errors in the source code which were found using funqual.  

\noindent\begin{minipage}[t]{\linewidth}
\begin{lstlisting}[caption={Output of running funqual on robotics library.  This is not the entire output, but rather a small snippet of it},label={lst:app:blocking:output}]
Rule violation: `slow` function indirectly called from `hi_freq` context
        Path:   lib/sensors/SPIGyro.cpp::frc973::SPIGyro::Run(void *) (60,18)
        -calls: lib/sensors/SPIGyro.cpp::frc973::SPIGyro::ReadPartID() (134,14)
        -calls: lib/sensors/SPIGyro.cpp::frc973::SPIGyro::DoRead(uint8_t) (92,14)
        -calls: lib/sensors/SPIGyro.cpp::frc973::SPIGyro::DoTransaction(uint32_t, uint32_t *) (128,10)
        -calls: stdio.h::printf(const char *__restrict, ...) (362,12)

Rule violation: `slow` function indirectly called from `hi_freq` context
        Path:   lib/sensors/SPIGyro.cpp::frc973::SPIGyro::Run(void *) (60,18)
        -calls: lib/sensors/SPIGyro.cpp::frc973::SPIGyro::InitializeGyro() (67,10)
        -calls: stdio.h)::printf(const char *__restrict, ...) (362,12)

Rule violation: `slow` function indirectly called from `hi_freq` context
        Path:   lib/sensors/SPIGyro.cpp::frc973::SPIGyro::Run(void *) (60,18)
        -calls: lib/sensors/SPIGyro.cpp::frc973::SPIGyro::InitializeGyro() (67,10)
        -calls: lib/sensors/SPIGyro.cpp::frc973::SPIGyro::DoTransaction(uint32_t, uint32_t *) (128,10)
        -calls: stdio.h::printf(const char *__restrict, ...) (362,12)
\end{lstlisting}
\end{minipage}

