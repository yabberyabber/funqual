Static analysis tools can aid programmers by reporting potential programming mistakes prior to the execution of a program.  Funqual is a static analysis tool that reads C++17 code ``in the wild'' and checks that the function call graph follows a set of rules which can be defined by the user.  This sort of analysis can help the programmer to avoid errors such as accidentally calling blocking functions in time-sensitive contexts or accidentally allocating memory in heap-sensitive environments.  To accomplish this, we create a type system whereby functions can be given user-defined type qualifiers and where users can define their own restrictions on the call graph based on these type qualifiers.  We demonstrate that this tool, when used with hand-crafted rules, can catch certain types of errors which commonly occur in the wild.  We claim that this tool can be used in a production setting to catch certain kinds of errors in code before that code is even run.  

